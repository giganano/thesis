
\section{Results}
\label{outflows:sec:results}

\subsection{Simplified Examples}
\label{outflows:sec:simplified-example}

We begin by comparing the fiducial model from Chapter~\ref{migration} with a
simplified model in which~$\eta = 0$ to highlight generic differences between
models with and without mass loading.
In Chapter~\ref{migration}, the strength of mass loading increases
exponentially with radius according to
\begin{equation}
\eta = \frac{\ycc{O}}{Z_{\text{O},\odot}}
e^{-\grad{O} (\ln 10) (R - R_\odot)} + r - 1.
\end{equation}
Under this parameterization, the equilibrium abundance~$Z_\text{O}$ scales with
radius according to the specified gradient~\grad{O}.
In the case of~$\eta = 0$, the stellar gradient instead arises out of a handful
of effects, including less efficient star formation (i.e., higher~$\tau_\star$)
and a more extended SFH (i.e., high~$\timescale{sfh}$) with increasing radius.
\par
We calibrate a model with~$\eta = 0$ everywhere such that the mode of the MDF
scales with radius according to the gradient we measured
in~\S~\ref{outflows:sec:empirical:gradients}.
The distribution in~$Z$ for some element can be expressed as
\begin{equation}
\frac{dN}{d \ln Z} = Z \frac{dN / dt}{dZ / dt}
\propto \frac{\dot{M}_\star}{(\dot{Z} / Z)}.
\end{equation}
By differentiating the above expression and making a handful of additional
chain rule substitutions, it is straightforward to show that
\begin{equation}
\frac{d^2N}{d \ln Z^2} \propto \dot{M}_\star
\left(\frac{Z}{\dot{Z}}\right)^2
\left(
\frac{\ddot{M}_\star}{\dot{M}_\star} + \frac{\dot{Z}}{Z} -
\frac{\ddot{Z}}{\dot{Z}}
\right).
\end{equation}
Herein lies part of the utility of the mode as the summary statistic when the
underlying model comes from GCE.



% The distribution in~$Z_\text{O}$ can be expressed as
% \begin{equation}
% \frac{dN}{d \ln Z_\text{O}} = Z_\text{O} \frac{dN/dt}{dZ_\text{O}/dt}
% \propto \frac{\dot{M}_\star}{(\dot{Z}_\text{O} / Z_\text{O})}.
% \end{equation}
% By differentiating the above and making a handful of additional chain rule
% substitutions, it is straightforward to show that
% \begin{equation}
% \frac{d^2N}{d \ln Z_\text{O}^2} = \dot{M}_\star
% \left(\frac{Z_\text{O}}{\dotZ{}})
% \end{equation}





% The O distribution can be expressed as
% \begin{equation}
% \frac{dN_\star}{d \ln Z_\text{O}} = Z_\text{O} \frac{dN_\star}{dZ_\text{O}}
% = Z_\text{O} \frac{\dot{N}_\star}{\dot{Z}_\text{O}}
% \propto \frac{\dot{M}_\star}{(\dot{Z}_\text{O} / Z_\text{O})}.
% \end{equation}
% That is, the MDF can be specified as a parametric function of time given the
% SFH and the solution to~$Z_\text{O}(t)$.
% \begin{equation}
% \frac{d^2N_\star}{d\ln Z_\text{O}^2} = 
% \end{equation}


% The MDF in~\oh~can be expressed as
% \begin{equation}
% \frac{dN}{d\oh} = \left(\ln 10\right)
% Z_\text{O} \frac{dN}{dZ_\text{O}}
% = \left(\ln 10\right) Z_\text{O} \frac{\dot{N}}{\dot{Z}_\text{O}}
% \propto \frac{\dot{M}_\star}{\left(\dot{Z}_\text{O} / Z_\text{O}\right)}
% \end{equation}



