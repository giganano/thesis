
\section{Results}
\label{outflows:sec:results}

\subsection{Simplified Examples}
\label{outflows:sec:simplified-example}

We begin by comparing the fiducial model from Chapter~\ref{migration} with a
simplified model in which~$\eta = 0$ to highlight generic differences between
models with and without mass loading.
In Chapter~\ref{migration}, the strength of mass loading increases
exponentially with radius according to
\begin{equation}
\eta = \frac{\ycc{O}}{Z_{\text{O},\odot}}
e^{-\grad{O} (\ln 10) (R - R_\odot)} + r - 1.
\end{equation}
Under this parameterization, the equilibrium abundance~$Z_\text{O}$ scales with
radius according to the specified gradient~\grad{O}.
In the case of~$\eta = 0$, the stellar gradient instead arises out of a handful
of effects, including less efficient star formation (i.e., higher~$\tau_\star$)
and a more extended SFH (i.e., high~$\timescale{sfh}$) with increasing radius.
\par
We calibrate a model with~$\eta = 0$ everywhere such that the mode of the MDF
scales with radius according to the gradient we measured
in~\S~\ref{outflows:sec:empirical:gradients}.
The distribution in~$Z$ for some element can be expressed as
\begin{equation}
\frac{dN}{d \ln Z} = Z \frac{dN / dt}{dZ / dt}
\propto \frac{\dot{M}_\star}{(\dot{Z} / Z)}.
\end{equation}
By differentiating the above expression and making a handful of additional
chain rule substitutions, it is straightforward to show that
\begin{equation}
\frac{d^2N}{d \ln Z^2} \propto \dot{M}_\star
\left(\frac{Z}{\dot{Z}}\right)^2
\left(
\frac{\ddot{M}_\star}{\dot{M}_\star} + \frac{\dot{Z}}{Z} -
\frac{\ddot{Z}}{\dot{Z}}
\right).
\end{equation}
Herein lies a portion of our motivation for choosing the mode as our summary
statistic in quantifying abundance gradients.
From the above expression, it is clear that the peak of the MDF is produced
when
\begin{equation}
\frac{\ddot{M}_\star}{\dot{M}_\star} = \frac{\ddot{Z}}{\dot{Z}} -
\frac{\dot{Z}}{Z},
\end{equation}
allowing one to solve for the time at which the peak is produced to determine
its position.
For simple SFHs, such as those in~\S~\ref{outflows:sec:gce:onezone:simple-cases},
the solution is straightforward since one can express~$Z_\text{O}(t)$
analytically.
\par
In the case of a single exponential SFH, differentiating
$\dot{M}_\star \propto e^{-t / \tau_\text{sfh}}$ and
$f_\text{sfh} = 1 - e^{-t / \taupsi{O}}$ (see
Table~\ref{outflows:tab:f-sfh-forms}) and solving for the time~$t_\text{max}$
at which the turnover in the MDF is produced yields
\begin{equation}
t_\text{max} = -\taupsi{O} \ln \left(1 - \frac{\timescale{sfh}}{\taupsi{O}}
\right).
\end{equation}
Computing~\oh~based on~$f_\text{sfh}(t_\text{max})$ indicates that the mode
occurs at
\begin{equation}
\text{mode([O/H])} = \log_{10}
\left(\frac{\ycc{O} \timescale{sfh}}{Z_{\text{O},\odot} \tau_\star}\right).
\end{equation}
\par
In principle, the mode will shift due to changes in both~$\tau_\star$ and
\timescale{sfh}.
Since this is a deliberately simplified example, we do not use the three
component power-law~$\dot{\Sigma}_\star - \Sigma_g$ relationship as in the
$\eta > 0$ comparison case (see discussion below).
We instead take a single power-law, enabling a straightforward solution
for~$\tau_\star$ and therefore~\timescale{sfh} as a function of radius.
By definition,
\begin{equation}
\begin{split}
\Sigma_g \tau_\star^{-1} &\propto \Sigma_g^N
\\
\tau_\star & \propto \Sigma_g^{1 - N}
\\
& \propto e^{(N - 1) R / R_g},
\end{split}
\end{equation}
where~$N$ is the power-law index of the~$\dot{\Sigma}_\star - \Sigma_g$
relation and~$R_g$ is the scale radius of the gas disk.
Combining terms, solving for~$\timescale{sfh}$, and replacing mode([O/H]) with
the desired abundance gradient yields the following relationship between
\timescale{sfh} and~$R$:
\begin{equation}
\tau_\text{sfh} = \tau_{\star,0} \frac{Z_{\text{O},\odot}}{\ycc{O}}
\exp\left[
(N - 1)\frac{R}{R_g} + \grad{O}(\ln 10)(R - R_\odot)
\right],
\end{equation}
where~$\tau_{\star,0}$ simply sets the value of~$\tau_\star$ at~$R = 0$,
$Z_{\text{O},\odot}$ is the O abundance in the Sun, and~$R_\odot = 8$ kpc is
the Galactocentric radius of the Sun.
We take~$\grad{O} = -0.06$ kpc$^{-1}$ from our measurements
in~\S~\ref{outflows:sec:empirical:gradients}.
We adopt~$N = 1.5$ based on the global SFRs and surface densities of
low-redshift star forming spirals~\citep{Kennicutt1998}.
Based on the presence of the central molecular zone in the inner few hundred
pc of the Galaxy~\citep[e.g.,][]{Morris1996, Dahmen1998, PiercePrice2000,
Hatchfield2020} and~\citeauthor{Leroy2008}'s~\citeyearpar{Leroy2008}
measurement of~$\tau_\star \approx 2$ Gyr for purely molecular gas, we
attribute this value to~$\tau_{\star,0}$.



% The distribution in~$Z_\text{O}$ can be expressed as
% \begin{equation}
% \frac{dN}{d \ln Z_\text{O}} = Z_\text{O} \frac{dN/dt}{dZ_\text{O}/dt}
% \propto \frac{\dot{M}_\star}{(\dot{Z}_\text{O} / Z_\text{O})}.
% \end{equation}
% By differentiating the above and making a handful of additional chain rule
% substitutions, it is straightforward to show that
% \begin{equation}
% \frac{d^2N}{d \ln Z_\text{O}^2} = \dot{M}_\star
% \left(\frac{Z_\text{O}}{\dotZ{}})
% \end{equation}





% The O distribution can be expressed as
% \begin{equation}
% \frac{dN_\star}{d \ln Z_\text{O}} = Z_\text{O} \frac{dN_\star}{dZ_\text{O}}
% = Z_\text{O} \frac{\dot{N}_\star}{\dot{Z}_\text{O}}
% \propto \frac{\dot{M}_\star}{(\dot{Z}_\text{O} / Z_\text{O})}.
% \end{equation}
% That is, the MDF can be specified as a parametric function of time given the
% SFH and the solution to~$Z_\text{O}(t)$.
% \begin{equation}
% \frac{d^2N_\star}{d\ln Z_\text{O}^2} = 
% \end{equation}


% The MDF in~\oh~can be expressed as
% \begin{equation}
% \frac{dN}{d\oh} = \left(\ln 10\right)
% Z_\text{O} \frac{dN}{dZ_\text{O}}
% = \left(\ln 10\right) Z_\text{O} \frac{\dot{N}}{\dot{Z}_\text{O}}
% \propto \frac{\dot{M}_\star}{\left(\dot{Z}_\text{O} / Z_\text{O}\right)}
% \end{equation}



