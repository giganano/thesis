
%%% Surveys
%
% Goddard et al. (2017), (MaNGA) in their abstract: "This result suggests that
% the merger history plays a relatively small role in shaping metallicity
% gradients of galaxies."



%%% Simulations
%
% Pilkington et al., 2012, A&A, 540, A56: suite of 25 disk in cosmological
% simulation from RAMSES evolve to self-similar abundance gradients, despite
% each evolving within the "inside-out" paradigm. Differences driven by the
% efficiency of star formation as a function of radius --> lines up almost
% perfectly with the equilibrium picture.

\section{Discussion and Conclusions}
\label{outflows:sec:disc-conc}

The evolution of the metallicity gradient traced by APOGEE favors GCE models in
which it arises due to a decrease in the equilibrium abundance with radius.
In this scenario, the equilibrium is set by the mass loading factor~$\eta$,
which describes the efficiency with which ISM gas is swept up and ejected due
to feedback.
We therefore expect variations in the gravitational potential with Galactic
region to establish this connection between~$\eta$ and radius.
\par
Classically interpreted as a consequence of inside-out galaxy growth
\citep[e.g.,][]{Kauffmann1996}, this interpretation reframes abundance
gradients as an equilibrium phenomenon whose slope traces the density profile
of the system.
The first principles model of~\citet{Sharda2021b} makes similar predictions.
Once it is near equilibrium, the ISM loses memory of its initial conditions,
and our APOGEE sample indicates this first occurred~$\sim$$9$ Gyr ago (see
discussion in~\S~{\color{red} X}).
The effects of inside-out galaxy growth on the MW abundance gradient should
therefore be locked up in age~$\gtrsim$ 10 Gyr stars and at
redshift~$z \gtrsim$ 1.5 for external galaxies.
\par
If this equilibrium model is accurate, then due to the relationship between
a galaxy's mass and its scale radius {\color{red} (citation)}, a secondary
correlation between mass and the slope of the metallicity gradient would be
expected.
Such a relationship is indeed observed, though there are indications that the
trend with mass may be complicated~\citep{Belfiore2017, Maiolino2019}.
However, the correlations are weak and the measurements are uncertain
\citep{Yuan2013, Acharyya2020, Poetrodjojo2021}, though perhaps difficulties
are not surprising if it is indeed a secondary correlation.
From a simulations perspective,~\citet{Pilkington2012} find that 25 disk
galaxies in cosmological simulations show similar abundance gradients, but
each evolved to this self-similarity differently, as expected if chemical
equilibrium quickly washes out the effects of the SFH.
Based on full spectrual fitting of MaNGA galaxies,~\citet{Goddard2017} also
argue that a galaxy's merger history plays a relatively small role in shaping
the metallicity gradient.


