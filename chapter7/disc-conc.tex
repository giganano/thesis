
%%% Surveys
%
% Goddard et al. (2017), (MaNGA) in their abstract: "This result suggests that
% the merger history plays a relatively small role in shaping metallicity
% gradients of galaxies."



%%% Simulations
%
% Pilkington et al., 2012, A&A, 540, A56: suite of 25 disk in cosmological
% simulation from RAMSES evolve to self-similar abundance gradients, despite
% each evolving within the "inside-out" paradigm. Differences driven by the
% efficiency of star formation as a function of radius --> lines up almost
% perfectly with the equilibrium picture.

\section{Discussion and Conclusions}
\label{outflows:sec:disc-conc}

We have characterized stellar age and abundance gradients in the MW as
traced by red giants in APOGEE.
As expected from inside-out growth in the~$\Lambda$CDM paradigm
\citep[e.g.,][]{White1991}, inner Galaxy stars tend to be old ($\sim$$6 - 10$
Gyr), and outer Galaxy stars tend to be young ($\sim$$2 - 6$ Gyr, see
Fig.~\ref{outflows:fig:age-xh-dists}).
In qualitative agreement with previous studies~\citep[e.g.,][]{Cheng2012,
Frinchaboy2013, Hayden2015, Myers2022}, characteristic stellar metallicities
decrease with increasing radius.
Distributions in both quantities are skew-negative in the inner Galaxy and
skew-positive in the outer Galaxy.
\par
The median stellar age is~$\sim$8 Gyr in the inner~$\sim$few kpc and decreases
steadily at a slope of~$\nabla\tau_{1/2} = -0.375 \pm 0.036$ Gyr/kpc.
At~$R \gtrsim 10$ kpc, the gradient flattens off and inverts sign slightly,
with the youngest median ages of~$\sim$$3.5$ Gyr occurring at~$\sim$$11 - 12$
kpc.
Stellar abundances span almost exactly one order of magnitude across the full
extent of the disk.
The O gradient at~$\grad{O} = -0.062 \pm 0.001$ kpc$^{-1}$ is slightly
shallower than that of Fe at~$\grad{Fe} = -0.070 \pm 0.003$ kpc$^{-1}$.
This measurement of~\grad{O} is also broadly consistent with measurements
derived from Cepheid variables~\citep{Luck2011a, Luck2011b, Genovali2014,
Lemasle2018}.
Our estimates of these gradients have small statistical uncertainties due to
the large sample size, though the APOGEE selection function introduces a
significant source of systematic uncertainty which we have not quantified
(see discussion in~\S~\ref{outflows:sec:empirical:caveats}).
\par
We then conditioned stars on their ages to measure the evolution in the
abundance gradient over time.
We find that a gradient closely resembling that of the present day was
established as long as~$\sim$9 Gyr ago in both \space \oh \space and \space \feh
\space across essentially the entire disk, with a positive slope in Fe and
flat in O in our oldest age bin ($9 - 10$ Gyr).
There are only small perturbations after the gradient was initially set (see
Fig.~\ref{outflows:fig:gradxh-fixed-age} and discussion
in~\S~\ref{outflows:sec:empirical}), the significance of which is unclear given
the systematic uncertainty of the APOGEE selection function.
If significant, the fluctuations likely reflect the detailed SFH of the disk,
in particular indicating an accretion event in the outer disk about~$\sim$6 Gyr
ago.
This result is broadly consistent with Imig et al.'s (2023, in prep.) dissection
of the stellar gradient with ages estimated from C/N ratios.
Having originally made these measurements with the~\textsc{AstroNN} APOGEE DR17
value added catalog~\citep{Mackereth2019b}, we successfully reproduced this
result with the more recent~\citet{Leung2023} age catalog, indicating that
systematic errors in the age and abundance determination, at least at the level
of differences of these two catalogs, do not impact our conclusion (see
discussion in~\S~\ref{outflows:sec:empirical:caveats}).
\par
This time evolution, or lack thereof, in the Galaxy's metallicity gradient
is readily explained qualitatively by GCE models that predict the ISM to reach
an equilibrium abundance, or ``saturate,'' early in its evolutionary history
(see discussion in~\S~\ref{outflows:sec:gce:predictions}).
GCE models that do not predict this saturation within the lifetime of the
Galactic disk instead predict a montonic rise in the metal abundances with time
at fixed radius.
Such an enrichment history results in broad metallicity distributions at all
radii and an abundance gradient whose normalization increases with time.
Based on this observational diagnostic, our APOGEE sample favors a scenario
in which the ISM has been at or near equilibrium for~$\sim$9 Gyr.
\par
This equilibrium interpretation is supported by both observational results and
simulation predictions.
\citet{Pilkington2012} demonstrate that a compilation of 25 disk galaxies from
cosmological simulations develop similar abundance gradients despite
significantly different evolutionary histories.
Based on full spectral fitting of MaNGA galaxies,~\citet{Goddard2017} also
argue that a galaxy's merger history plays a relatively small role in shaping
the abundance gradient.
\par
Classically interpreted as a consequence of inside-out Galaxy growth
\citep{Kauffmann1996}, this equilibrium interpretation is a significant shift
in the discourse surrounding abundance gradients.
In principle, there are many ways that GCE models can predict the ISM
abundances to reach equilibrium early in the Galaxy's evolutionary history.
Under the scenarios discussed in~\S~\ref{outflows:sec:gce:calibration}, we
would expect the slope of a spiral galaxy's abundance gradient to trace the
slope of its gravitational potential well.
We clarify that these results do not dispute inside-out galaxy growth; once
equilibrium is reached, GCE models suggest that the ISM abundance simply loses
memory of its initial conditions~\citep[see, e.g.,][]{Weinberg2017b}.
As a result, this interpretation would suggest that the effects of inside-out
growth in the age-abundance structure of the Galaxy are locked up
in~$\gtrsim$9 Gyr old stars.

% This time evolution -- or lack thereof -- in the Galaxy's metallicity gradient
% is readily explained qualitatively by GCE models which invoke equilibrium
% abundances.
% In this scenario, solar and even super-solar abundances can be achieved within
% the first couple Gyrs of chemical enrichment, after which there is little to no
% evolution~\citep[e.g.,][]{Larson1972, Weinberg2017b}.
% The gradient itself arises in this scenario out of changes in the
% equilibrium abundance as in Chapter~\ref{migration}, though this detail of that
% model was assumed rather than deduced from data.
% GCE models in the literature in which abundances do not reach an equilibrium
% instead predict an upward evolution in the normalization of the gradient with
% time (see, e.g., Fig. 2 of~\citealt{Minchev2013}).
% \par
% The O gradient we measure is also consistent within~$1\sigma$
% with~\citet{MendezDelgado2022}'s \citet{MendezDelgado2022} measurement in the
% gas-phase, in which they carefully accounted for the temperature
% inhomogeneities within HII regions.
% This agreement is expected under the equilibrium scenario~\citep{Weinberg2017b}.
% The decrease in abundances in the outer disk~$\sim$6 Gyr ago, if significant,
% is readily explained by dilution due to accretion of metal-poor gas.
% In this scenario, the infalling gas dilutes the ISM metallicity, after which O
% and Fe re-enrich on different timescales in the ensuing burst
% in star formation (e.g., Chapter~\ref{bursts}).
% \par
% Classically interpreted as a consequence of inside-out Galaxy growth
% \citep{Kauffmann1996}, this equilibrium interpretation is a significant
% challenge to the understanding of abundance gradients.
% In the Chapter~\ref{migration} scenario, ISM gas is more readily swept up due
% to feedback and ejected in an outflow with increasing radius.
% It is also possible that a significant portion of SN ejecta is lost directly to
% an outflow~\citep{Dalcanton2007, Peeples2011, Christensen2018, Chisholm2018,
% Cameron2021}, and this fraction could in principle decrease with radius in a
% corresponding fashion.
% Under either interpretation, we expect the slope of a spiral galaxy's
% abundance gradient to trace the slope of its gravitational potential well.
% \par
% We clarify that this is not a rebuttle against inside-out galaxy growth.
% The evidence suggests that most of the Galactic disk reached an approximate
% equilibrium abundance as long as 9 Gyr ago, after which GCE models would
% suggest that it simply loses memory of its initial conditions (see discussion
% in, e.g.,~\citealt{Weinberg2017b}).
% We infer that the effects of inside-out growth are simply washed away and
% locked up in~$\gtrsim$9 Gyr old stars.
% This argument would suggest that the detailed SFH should play a minimal role
% in shaping the gradient at late times, which is both predicted from simulations
% and supported observationally.
% \citet{Pilkington2012} demonstrate that a compilation of 25 disk galaxies from
% cosmological simulations develop similar abundance gradients despite
% significantly different evolutionary histories.
% Based on full spectral fitting of MaNGA galaxies,~\citet{Goddard2017} also
% argue that a galaxy's merger history plays a relatively small role in shaping
% the abundance gradient.

