
%%% Stellar gradients values in the literature
%
% Myers+22: -0.073 +/- 0.002 dex/kpc in [Fe/H] from open clusters in APOGEE
% 	- OCCAM survey
%
% Frinchaboy+13: -0.09 +/- 0.03 dex/kpc in [Fe/H] in APOGEE
% 	- also OCCAM
%
% Hayden+14: Evidence that it flattens within ~6 kpc, also find vertical
% gradients whose slope depend on radius. -0.087 +/- 0.002 dex/kpc in the plane
% and outside 6 kpc. SDSS-III/APOGEE DR10
%
% Nordstroem+04: -0.099 +/- 0.011 dex/kpc for intermediate age (4-6 Gyr)
% populations in GCS (Geneva-Copenhagen Survey), -0.076 +/- 0.014 for young
% populations (<1.5 Gyr).
%
% Cheng+12: -0.066 +0.030/-0.044 dex/kpc from SEGUE in the midplane.
%
% Weinberg+19: find -0.06 dex/kpc in [Mg/H] for |z| < 0.05 kpc
%
% Anders+17: constant at -0.07 dex/kpc between age = 1-4 Gyr, and slightly
% flatter, in agreement with Cepheid results, for the <1 Gyr old populations.
% Older ages reach values compatible with a flat distribution.
% Data: CoRoT-APOGEE combining APOGEE abundances with 606 solar-like red-giants
% with asteroseismic observations availabe through CoRoT.
%
% Maciel & Quireza 99: extrapolate from PNe to estimate temporal flattening of
% [O/H] gradient with age (-0.004 dex/kpc/Gyr). --> constant slope in the
% inner disk but some flattening in the outer disk --> we see the same thing
% in our sample.


%%% Gas gradients in the literature
%
% C. Esteban, et al., 2022, ApJ, 931, 92
% J.E. Mendez-Delgado, et al., 2022, MNRAS, 510, 4436
% J.E. Mendez-Delgado, et al., 2023, Nature, 618, 249
% J.P. Simspon, et al., 1995, ApJ, 444, 721 -> far infrared lines from HII
% 	regions.
% Afflerbach, Churchwell & Werner, 1997, ApJ, 478, 190 -> fine structure lines
% 	compact HII regions.
% 



%%% Simulations
%
% Pilkington et al., 2012, A&A, 540, A56: suite of 25 disk in cosmological
% simulation from RAMSES evolve to self-similar abundance gradients, despite
% each evolving within the "inside-out" paradigm. Differences driven by the
% efficiency of star formation as a function of radius --> lines up almost
% perfectly with the equilibrium picture.






% Minchev & Famaey 2010: radial mixing produces skew in MDFs (?)

\section{Introduction}
\label{outflows:sec:intro}
\noindent
Radial metallicity gradients are ubiquitous in spiral galaxies.
Within the Milky Way (hereafter MW), direct measurements are possible with
star-by-star abundances.
Typical results indicate slopes ranging from~$-0.06$ dex/kpc to somewhat
steeper values of~$-0.1$ dex/kpc~\citep[e.g.,][]{Nordstroem2004, Cheng2012,
Frinchaboy2013, Hayden2014, Weinberg2019, Myers2022}.
% The metallicity-dependent ratio of the first overtone and fundamental periods
% of mixed-mode Cepheids indicate slopes broadly consistent with the shallower
% value of~$-0.06$ dex/kpc inferred from direct abundance measurements
% \citep{Luck2011a, Luck2011b, Genovali2014, Lemasle2018}.
The interestingly diverse range of inferred slopes arises for a variety of
reasons, including but not limited to the selection function of the survey,
systematics introduced by its data reduction pipeline, the age range of the
sample, and selection criteria in radius and mid-plane
distance~\citep[see discussion in, e.g.,~\S~5 of][]{Anders2017}.
\par
Measurements of metal abundances in HII regions, both in the
MW~\citep[e.g.,][]{Simpson1995, Afflerbach1997, Esteban2022, MendezDelgado2022,
MendezDelgado2023} and in surveys of external galaxies
\citep[e.g.,][]{Belfiore2017, Berg2020, Franchetto2021, Boardman2022}, indicate
similarly sloped radial gradients.
These variations in metallicity within galaxies have classically been
interpreted as evidence for ``inside-out'' disk formation, whereby the outer
regions form later and on longer timescales than the inner disk (see, e.g.,
recent review articles by~\citealt{Kewley2019, Maiolino2019}, and
\citealt{Sanchez2020}).
In this chapter, we demonstrate with galactic chemical evolution (GCE) models
that the origin of the abundance gradient in the MW is closely related to the
parameterization of mass-loaded outflows from the disk.


% \begin{itemize}

% 	\item Radial abundance gradients are ubiquitous in spiral galaxies
% 	\citep[e.g.,][]{Simpson1995, Afflerbach1997, Molla1997, Molla1999,
% 	Belfiore2017, Franchetto2021, Boardman2022}.

% 	\item Within the MW, direct measurements are possible with star-by-star
% 	abundances, though the results in the literature are interestingly diverse
% 	For example,~\citet{Nordstroem2004} find that young stellar populations
% 	($<1.5$ Gyr) in the Geneva-Copenhagen Survey follow a gradient of
% 	$\grad{Fe} = -0.076$ kpc$^{-1}$, while intermediate age stars ($4 - 6$ Gyr)
% 	have a slightly steeper gradient of~$\grad{Fe} = -0.099$ kpc$^{-1}$.
% 	More recently, measurements in open clusters suggest
% 	$\grad{Fe} \approx -0.07$ dex/kpc~\citep{Frinchaboy2013, Myers2022}.
% 	\citet{Hayden2014} find a relatively flat gradient within~$R < 6$ kpc, with
% 	a slope of~$\grad{Fe} = -0.087$ kpc$^{-1}$ at larger radii.
% 	Variations in the inferred slopes can arise for a number of reasons,
% 	including but not limited to the selection criteria in radius and mid-plane
% 	distance, the age range of the sample, the selection function of the survey,
% 	and the systematics introduced by its data reduction pipeline (see
% 	discussion in, e.g.,~\S~5 of~\citealt{Anders2017}).

% 	\item The metallicity-dependent ratio of the first overtone and fundamental
% 	periods of mixed-mode Cepheids provides an additional tracer of radial
% 	abundance gradients.
% 	In this manner,~\citet{Luck2011a},~\citet{Luck2011b},
% 	and~\citet{Genovali2014} find slopes broadly consistent with~$-0.06$
% 	kpc$^{-1}$, though~\citet{Lemasle2018} infer a slightly shallower slope
% 	of~$-0.04$ kpc$^{-1}$.

% 	\item Negatively sloped abundance gradients have classically been
% 	interpreted as evidence for ``inside-out'' disk formation, whereby the
% 	outer regions form later and on longer timescales than the inner disk
% 	(see, e.g., recent review articles by~\citealt{Kewley2019, Maiolino2019},
% 	and~\citealt{Sanchez2020}).
	
% \end{itemize}


% Points to hit in the introduction:
% \begin{itemize}

% 	\item Radial abundance gradients are ubiquitous in Milky Way-like galaxies.
% 	Direct measurements from APOGEE (there are some refs in the migration
% 	paper) and other abundance catalogs.
% 	Can also be measured from the metallicity-dependent ratio of the first
% 	overtone and fundamental periods of mixed-mode Cepheids.
% 	\citet{Luck2011a},~\citet{Luck2011b}, and~\citet{Genovali2014} find slopes
% 	that are broadly consistent with~$-0.06$ dex/kpc from these data, though
% 	\citet{Lemasle2018} infer a slightly lower value of~$-0.04$ dex/kpc.
% 	The quantitative differences in the results from different methods are
% 	interesting and potentially reflect systematic uncertainties, though it is
% 	well understood that stellar abundances span~$\sim$1 order of magnitude
% 	over the full extent of the Galactic disk.

% 	\item Despite the attention that the abundance gradient has received in the
% 	literature, its origin is poorly understood.
% 	{\color{red} $\sim$few sentences:} Hit some of the main citations on this
% 	topic and summarize their primary results -- e.g.,~\citet{Sharda2021a,
% 	Sharda2021b} on gas-phase gradients and the physics thereof (accretion,
% 	advection, etc.);~\citet{Minchev2013, Minchev2014} on the stellar gradient
% 	and the notion of inside-out Galaxy formation being the main driver.

% 	\item Here we demonstrate with chemical evolution models that the origin
% 	of the gradient and its evolution with time are closely related to the
% 	strength of mass-loading in outflows from the Galactic disk.
% 	Draw on some of the text from Appendix B of the dwarf galaxy archaeology
% 	paper to provide an overview of some of the different prescriptions for
% 	outflows and mass-loading in the literature.

% 	\item Strength of mass-loading is strongly degenerate with the
% 	normalization of elemental yields, making empirical determinations via
% 	GCE models difficult.

% \end{itemize}

