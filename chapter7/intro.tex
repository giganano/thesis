
\section{Introduction}
\label{outflows:sec:intro}
\noindent
Points to hit in the introduction:
\begin{itemize}

	\item Radial abundance gradients are ubiquitous in Milky Way-like galaxies.
	Direct measurements from APOGEE (there are some refs in the migration
	paper) and other abundance catalogs.
	Can also be measured from the metallicity-dependent ratio of the first
	overtone and fundamental periods of mixed-mode Cepheids.
	\citet{Luck2011a},~\citet{Luck2011b}, and~\citet{Genovali2014} find slopes
	that are broadly consistent with~$-0.06$ dex/kpc from these data, though
	\citet{Lemasle2018} infer a slightly lower value of~$-0.04$ dex/kpc.
	The quantitative differences in the results from different methods are
	interesting and potentially reflect systematic uncertainties, though it is
	well understood that stellar abundances span~$\sim$1 order of magnitude
	over the full extent of the Galactic disk.

	\item Despite the attention that the abundance gradient has received in the
	literature, its origin is poorly understood.
	{\color{red} $\sim$few sentences:} Hit some of the main citations on this
	topic and summarize their primary results -- e.g.,~\citet{Sharda2021a,
	Sharda2021b} on gas-phase gradients and the physics thereof (accretion,
	advection, etc.);~\citet{Minchev2013, Minchev2014} on the stellar gradient
	and the notion of inside-out Galaxy formation being the main driver.

	\item Here we demonstrate with chemical evolution models that the origin
	of the gradient and its evolution with time are closely related to the
	strength of mass-loading in outflows from the Galactic disk.
	Draw on some of the text from Appendix B of the dwarf galaxy archaeology
	paper to provide an overview of some of the different prescriptions for
	outflows and mass-loading in the literature.

	\item Strength of mass-loading is strongly degenerate with the
	normalization of elemental yields, making empirical determinations via
	GCE models difficult.

\end{itemize}

