
%%% Stellar gradients values in the literature
%
% Myers+22: -0.073 +/- 0.002 dex/kpc in [Fe/H] from open clusters in APOGEE
% 	- OCCAM survey
%
% Frinchaboy+13: -0.09 +/- 0.03 dex/kpc in [Fe/H] in APOGEE
% 	- also OCCAM
%
% Hayden+14: Evidence that it flattens within ~6 kpc, also find vertical
% gradients whose slope depend on radius. -0.087 +/- 0.002 dex/kpc in the plane
% and outside 6 kpc. SDSS-III/APOGEE DR10
%
% Nordstroem+04: -0.099 +/- 0.011 dex/kpc for intermediate age (4-6 Gyr)
% populations in GCS (Geneva-Copenhagen Survey), -0.076 +/- 0.014 for young
% populations (<1.5 Gyr).
%
% Cheng+12: -0.066 +0.030/-0.044 dex/kpc from SEGUE in the midplane.
%
% Weinberg+19: find -0.06 dex/kpc in [Mg/H] for |z| < 0.05 kpc
%
% Anders+17: constant at -0.07 dex/kpc between age = 1-4 Gyr, and slightly
% flatter, in agreement with Cepheid results, for the <1 Gyr old populations.
% Older ages reach values compatible with a flat distribution.
% Data: CoRoT-APOGEE combining APOGEE abundances with 606 solar-like red-giants
% with asteroseismic observations availabe through CoRoT.
%
% Maciel & Quireza 99: extrapolate from PNe to estimate temporal flattening of
% [O/H] gradient with age (-0.004 dex/kpc/Gyr). --> constant slope in the
% inner disk but some flattening in the outer disk --> we see the same thing
% in our sample.


%%% Gas gradients in the literature
%
% C. Esteban, et al., 2022, ApJ, 931, 92
% J.E. Mendez-Delgado, et al., 2022, MNRAS, 510, 4436
% J.E. Mendez-Delgado, et al., 2023, Nature, 618, 249
% J.P. Simspon, et al., 1995, ApJ, 444, 721 -> far infrared lines from HII
% 	regions.
% Afflerbach, Churchwell & Werner, 1997, ApJ, 478, 190 -> fine structure lines
% 	compact HII regions.
% 



%%% Simulations
%
% Pilkington et al., 2012, A&A, 540, A56: suite of 25 disk in cosmological
% simulation from RAMSES evolve to self-similar abundance gradients, despite
% each evolving within the "inside-out" paradigm. Differences driven by the
% efficiency of star formation as a function of radius --> lines up almost
% perfectly with the equilibrium picture.






% Minchev & Famaey 2010: radial mixing produces skew in MDFs (?)

\section{Introduction}
\label{outflows:sec:intro}
\noindent
Radial metallicity gradients are ubiquitous in spiral galaxies.
Within the Milky Way (hereafter MW), direct measurements are possible with
star-by-star abundances.
Typical results indicate slopes ranging from~$-0.06$ dex/kpc to somewhat
steeper values of~$-0.1$ dex/kpc~\citep[e.g.,][]{Nordstroem2004, Cheng2012,
Frinchaboy2013, Hayden2014, Weinberg2019, Myers2022}.
% The metallicity-dependent ratio of the first overtone and fundamental periods
% of mixed-mode Cepheids indicate slopes broadly consistent with the shallower
% value of~$-0.06$ dex/kpc inferred from direct abundance measurements
% \citep{Luck2011a, Luck2011b, Genovali2014, Lemasle2018}.
The interestingly diverse range of inferred slopes arises for a variety of
reasons, including but not limited to the selection function of the survey,
systematics introduced by its data reduction pipeline, the age range of the
sample, and selection criteria in radius and mid-plane
distance~\citep[see discussion in, e.g.,~\S~5 of][]{Anders2017}.
\par
Measurements of metal abundances in the gas-phase, both in the MW
\citep[e.g.,][]{Simpson1995, Afflerbach1997, Esteban2022, MendezDelgado2022,
MendezDelgado2023} and in external galaxies~\citep[e.g.,][]{Belfiore2017,
Berg2020, Franchetto2021, Lutz2021, Boardman2022}, indicate qualitatively
similar results.
In general, abundances in the MW and other spiral galaxies decrease by about
an order of magnitude or slightly more over the full extent of their disks.
These abundance variations within galaxies have classically been interpreted as
evidence for ``inside-out'' disk formation, whereby the outer regions form
later and on longer timescales than the inner disk (see, e.g., recent review
articles by~\citealt{Kewley2019, Maiolino2019}, and~\citealt{Sanchez2020}).
In this chapter, we demonstrate with galactic chemical evolution (GCE) models
that the origin and evolution of the abundance gradient in the MW is closely
related to the strength of mass loading in outflows from the Galactic disk.
\par
With regard to this parameter, GCE models in the literature typically fall into
one of two categories.
The publicly available GCE codes~\textsc{Omega}~\citep{Cote2017} and
\textsc{FlexCE}~\citep{Andrews2017} implement a linear relationship between the
mass outflow rate and the star formation rate (SFR;
i.e.,~$\dot{M}_\text{out} = \eta \dot{M}_\star$).
\vice\space(see Chapter~\ref{bursts}) implements the same parameterization.
Other GCE models omit mass loading
\citep[i.e.,~$\eta = 0$; e.g.,][]{Minchev2013, Minchev2014, Spitoni2019,
Spitoni2020, Spitoni2021, Gjergo2023}.



% (see discussion in Appendix~\ref{dga:sec:degeneracy})

