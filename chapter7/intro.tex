
%%% Stellar gradients values in the literature
%
% Myers+22: -0.073 +/- 0.002 dex/kpc in [Fe/H] from open clusters in APOGEE
% 	- OCCAM survey
%
% Frinchaboy+13: -0.09 +/- 0.03 dex/kpc in [Fe/H] in APOGEE
% 	- also OCCAM
%
% Hayden+14: Evidence that it flattens within ~6 kpc, also find vertical
% gradients whose slope depend on radius. -0.087 +/- 0.002 dex/kpc in the plane
% and outside 6 kpc. SDSS-III/APOGEE DR10
%
% Nordstroem+04: -0.099 +/- 0.011 dex/kpc for intermediate age (4-6 Gyr)
% populations in GCS (Geneva-Copenhagen Survey), -0.076 +/- 0.014 for young
% populations (<1.5 Gyr).
%
% Cheng+12: -0.066 +0.030/-0.044 dex/kpc from SEGUE in the midplane.
%
% Weinberg+19: find -0.06 dex/kpc in [Mg/H] for |z| < 0.05 kpc
%
% Anders+17: constant at -0.07 dex/kpc between age = 1-4 Gyr, and slightly
% flatter, in agreement with Cepheid results, for the <1 Gyr old populations.
% Older ages reach values compatible with a flat distribution.
% Data: CoRoT-APOGEE combining APOGEE abundances with 606 solar-like red-giants
% with asteroseismic observations availabe through CoRoT.
%
% Maciel & Quireza 99: extrapolate from PNe to estimate temporal flattening of
% [O/H] gradient with age (-0.004 dex/kpc/Gyr). --> constant slope in the
% inner disk but some flattening in the outer disk --> we see the same thing
% in our sample.


%%% Gas gradients in the literature
%
% C. Esteban, et al., 2022, ApJ, 931, 92
% J.E. Mendez-Delgado, et al., 2022, MNRAS, 510, 4436
% J.E. Mendez-Delgado, et al., 2023, Nature, 618, 249
% J.P. Simspon, et al., 1995, ApJ, 444, 721 -> far infrared lines from HII
% 	regions.
% Afflerbach, Churchwell & Werner, 1997, ApJ, 478, 190 -> fine structure lines
% 	compact HII regions.

% Minchev & Famaey 2010: radial mixing produces skew in MDFs (?)

\section{Introduction}
\label{outflows:sec:intro}
Radial metallicity gradients are ubiquitous in spiral galaxies.
Within the Milky Way (hereafter MW), direct measurements are possible with
star-by-star abundances.
Typical results indicate slopes ranging from~$\sim$$-0.06$ dex/kpc to somewhat
steeper values of~$\sim$$-0.1$ dex/kpc~\citep[e.g.,][]{Nordstroem2004,
Cheng2012, Frinchaboy2013, Hayden2014, Weinberg2019, Myers2022}.
As a result, characteristic abundances span about an order of magnitude across
the full extent of the disk, depending on the slope and the specific element.
Spectroscopic surveys of low-redshift spirals indicate that the Milky Way is
typical in this regard.
For example,~\citet{Goddard2017} and~\citet{Parikh2021} find qualitatively
similar stellar abundance profiles in MaNGA~\citep{Bundy2015}.
\par
Abundance measurements in Galactic HII regions indicate that gas-phase
metallicities also decline with radius~\citep[e.g.,][]{Simpson1995,
Afflerbach1997, Esteban2022, MendezDelgado2022, MendezDelgado2023}.
Once again, the interstellar medium (ISM) in the MW appears qualitatively
similar to other spirals~\citep[e.g.,][]{Belfiore2017, Berg2020,
Franchetto2021, Lutz2021, Boardman2022}.
These abundance variations with radius have classically been interpreted as
evidence for ``inside-out'' disk formation, whereby the outer regions form
later and on longer timescales than the inner disk (see, e.g.,
\citealt{Kauffmann1996}).
\par
In this chapter, we quantify the evolution of the abundance gradient in the
Milky Way over time by conditioning on stellar age.
We demonstrate that a metallicity gradient closely resembling that of young
stars was established as far back as~$\sim$9 Gyr ago, after which we find only
small perturbations in the slope and normalization.
From the perspective of galactic chemical evolution (GCE) models, this
approximate time-independence is indicative of chemical equilibrium in the
interstellar medium (ISM), whereby metal production by stars is balanced by
losses to various sinks, such as new star formation, outflows, and radial gas
flows~\citep{Larson1974, Weinberg2017b}.
Under this interpretation, inside-out growth plays a significant role in
shaping the origin of the gradient for only the first~$\sim$few Gyr of Galaxy
evolution.
\par
We present our sample and empirically quantify abundance and age gradients
in~\S~\ref{outflows:sec:empirical} below.
We describe our chemical evolution models and their predictions
in~\S~\ref{outflows:sec:gce}.
We summarize our conclusions and discuss their implications
in~\S~\ref{outflows:sec:disc-conc}.

% In this chapter, we demonstrate with galactic chemical evolution (GCE) models
% that the origin and evolution of the abundance gradient in the MW is closely
% related to the strength of mass loading in outflows from the Galactic disk.
% With regard to this parameter, GCE models in the literature typically fall into
% one of two categories.
% The publicly available GCE codes~\textsc{Omega}~\citep{Cote2017} and
% \textsc{FlexCE}~\citep{Andrews2017} implement a linear relationship between the
% mass outflow rate and the star formation rate (SFR;
% i.e.,~$\dot{M}_\text{out} = \eta \dot{M}_\star$).
% In Chapter~\ref{migration}, we parameterized the mass loading factor as
% exponential growth with radius. 
% Other GCE models haved omitted mass loading
% \citep[i.e.,~$\eta = 0$; e.g.,][]{Minchev2013, Minchev2014, Spitoni2019,
% Spitoni2020, Spitoni2021} based on the argument that metals ejected
% from the disk are re-accreted on short timescales and arrive at radii similar
% to where they formed~\citep{Melioli2008, Melioli2009, Spitoni2008,
% Spitoni2009}.
% \par
% Distinguishing between these classes of models is difficult, because the
% strength of mass-loading is strongly degenerate with the normalization of
% nucleosynthetic yields (see discussion in, e.g.,
% Appendix~\ref{dga:sec:degeneracy}).
% The two are the primary source and sink terms in computing enrichment rates in
% GCE models.
% Consequently, a model with high yields and strong mass loading generally
% has a counterpart model with lower yields and weak or no mass loading that
% makes similar predictions.
% Although observations of galactic winds provide evidence of mass loading
% \citep[see, e.g., the review by][]{Veilleux2020}, with current
% instrumentation the measurements are conclusive only for relatively extreme
% starburst systems (e.g., in M82,~\citealt{Lopez2020}; in NGC 253,
% \citealt{Lopez2023}; in Mrk 1486,~\citealt{Cameron2021}).
% It remains unclear if the same arguments can be extended to the MW where the
% surface density of star formation is considerably lower.
% \par
% Measurements of the deuterium abundance~\citep{Linsky2006, Prodanovic2010} and
% the~$^3$He/$^4$He ratio~\citep{Balser2018} in the local ISM suggest values
% close to that of primordial gas.
% \citet{Weinberg2017a} and~\citet{Cooke2022} argue that this result is best
% explained by GCE models with strong mass loading.
% However, abundances of these isotopes are challenging to measure, and in the
% case of deuterium, dust depletion may play a significant
% role~\citep{Romano2006}.
% Pinning down the solution with stellar models is difficult as well, because the
% yields are sensitive to many poorly understood processes (e.g., massive star
% explodability, rotationally induced mixing, convection and convective
% boundaries, mass loss prescriptions; see discussion in,
% e.g.,~\citealt{Romano2010, Griffith2021b, Gil-Pons2022}, and
% Chapter~\ref{ohno}).
% Even with accurate stellar yields, a significant portion of SN ejecta may be
% lost directly to a hot outflow, lowering effective yields in a manner
% independent of stellar physics~\citep{Dalcanton2007, Peeples2011,
% Christensen2018, Chisholm2018, Cameron2021}.
% \par
% Here, we demonstrate that these two classes of GCE models produce a MW-like
% abundance gradient on different timescales.
% In particulary, changes in the normalization of the gradient with stellar age
% are a feasible empirical diagnostic.
% We first quantify the total age and stellar abundance gradients in the Galaxy
% in~\S~\ref{outflows:sec:empirical}.
% In~\S~\ref{outflows:sec:gce}, we construct GCE models incorporating the effects
% of stellar migration and radial gas flows to assess the differences
% between~$\eta = 0$ and~$\eta > 0$.
% We then quantify the differences between the two and compare their predictions
% to our observed sample in~\S~\ref{outflows:sec:results}.
% We discuss our results and their implications
% in~\S~\ref{outflows:sec:disc-conc}.

% The metallicity-dependent ratio of the first overtone and fundamental periods
% of mixed-mode Cepheids indicate slopes broadly consistent with the shallower
% value of~$-0.06$ dex/kpc inferred from direct abundance measurements
% \citep{Luck2011a, Luck2011b, Genovali2014, Lemasle2018}.


% (see discussion in Appendix~\ref{dga:sec:degeneracy})

