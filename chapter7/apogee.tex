
\section{Empirical Age and Metallicity Gradients}
\label{outflows:sec:empirical}

\subsection{The Sample}
\label{outflows:sec:empirical:apogee}
There are many spectroscopic surveys to choose from to characterize the
abundance structure of the Galactic disk, such as
LAMOST~\citep{Luo2015}, GALAH~\citep{DeSilva2015, Martell2017},~\gaia-ESO
\citep{Gilmore2012}, and APOGEE\space\citep{Majewski2017}.
APOGEE is particularly well suited to this task, because it targets luminous
evolved stars accessible at large distances and observes them at near-IR
wavelengths, which are less susceptible to dust obscuration.
The disk sample is dominated by stars with Two Micron All Sky Survey
\citep{Skrutskie2006} magnitudes of~$7 < H < 13.8$ on a grid of sightlines at
Galactic latitudes of~$b = 0$,~$\pm 4^\circ$, and~$\pm 8^\circ$; targeting is
described in detail by~\citet{Zasowski2013, Zasowski2017},~\citet{Beaton2021},
and~\citet{Santana2021}.
APOGEE collects high-resolution spectra ($R \sim$ 22,500) between~$1.51$
and~$1.70~\mu$m~\citep{Wilson2019} on the 2.5 m Sloan Foundation Telescope
\citep{Gunn2006} at Apache Point Observatory and the 2.5 m du Pont
Telescope~\citep{Bowen1973} at Las Campanas Observatory, which are then
reduced and calibrated using the APOGEE data processing pipeline
\citep{Nidever2015}.
Stellar parameters such as~$T_\text{eff}$,~$\log g$, and abundances of 15 or
more elements per target star are then computed with the APOGEE Stellar
Parameters and Chemical Abundances Pipeline~\citep[ASPCAP;][]{Holtzman2015,
GarciaPerez2016}.
\par
In this chapter, we select stars from the seventeenth data
release~\citep[DR17;][]{Abdurrouf2022} that satisfy the following criteria:
\begin{itemize}

	\item \texttt{EXTRATARG == 0}

	\item \texttt{STAR\_BAD == 0}

	\item S/N~$\geq 80$

	\item $\log g = 1 - 3.8$

	\item $T_\text{eff} = 3500 - 5500$ K

\end{itemize}
We additionally exclude stars with surface gravities of~$\log g > 3$ and
$T_\text{eff} < 4000$ K to remove main sequence stars, yielding a sample
of 191,173 red giant and red clump stars.
\par
To constrain the evolution of the abundance gradient over time, we also need
star-by-star ages.
To this end, we make use of two age catalogs, both produced by deep
learning algorithms available through~\textsc{AstroNN}~\citep{Leung2019}.
\citet{Mackereth2019b} trained a Bayesian convolutional neural network on
APOGEE spectra and asteroseismic data from the APOKASC-2
catalog~\citep{Pinsonneault2018}.
Using a variational encoder-decoder algorithm~\citep[e.g.,][]{LeCun2015},
\citet{Leung2023} instead compress the APOGEE spectra and the asteroseismic
power spectra into lower-dimensional representations of themselves (i.e., a
\textit{latent space}).
They then train a modified random forest algorithm to predict the
asteroseismic power spectrum from the APOGEE spectra in this latent space.
While~\citet{Mackereth2019b} infer ages for the entire APOGEE catalog,
\citet{Leung2023} only do so for the range of surface gravities spanned by
their training sample ($\log g = 2.5 - 3.6$).

