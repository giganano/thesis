
\chapter{Normalizing the Star Formation History} 
\label{migration:sec:normalize_sfh} 

In this appendix, we derive a prescription for calculating the prefactors of an 
adopted star formation history (SFH) for each annulus in our models. As 
mentioned in~\S~\ref{migration:sec:methods:sfhs}, this procedure requires a unitless 
description of the time-dependence of the SFH in each annulus, denoted 
$f(t|R_\text{gal})$, and a unitless description of the radial dependence 
of the stellar surface density, denoted~$g(R_\text{gal})$. 
By additionally selecting a 
total stellar mass of the present day model Galaxy, the solution to the 
detailed form of the star formation history 
$\dot{\Sigma}_\star(t, R_\text{gal})$ is unique. With this approach, we assume 
that stellar migration does not significantly impact the form of 
$g(R_\text{gal})$, an assumption we demonstrate to be accurate 
in~\S~\ref{migration:sec:methods:surface_density_gradient}. 
\par 
The surface density of star formation with units of mass per area per time can 
be expressed in terms of~$f(t|R_\text{gal})$ as: 
\begin{equation} 
\dot{\Sigma}_\star(t, R_\text{gal}) = \dot{\Sigma}_\star(t = 0, R_\text{gal}) 
f(t|R_\text{gal}) 
\label{migration:eq:sfh_terms_of_f} 
\end{equation} 
and the present-day radial surface density gradient with units of mass per area 
as: 
\begin{equation} 
\Sigma_\star(R_\text{gal}) = \Sigma_\star(R_\text{gal} = 0) g(R_\text{gal}). 
\label{migration:eq:sigma_terms_of_g}  
\end{equation} 
The integral of~$\dot{\Sigma}_\star$ with time should yield the surface density 
gradient at a given radius~$R_\text{gal}$, up to a prefactor accounting for the 
return of stellar envelopes to the interstellar medium (ISM): 
\begin{subequations}\begin{align} 
\Sigma_\star(R_\text{gal}) &= (1 - r)\int_0^T \dot{\Sigma}_\star(t, 
R_\text{gal}) dt 
\\ 
&= (1 - r) \dot{\Sigma}_\star(t = 0, R_\text{gal})\int_0^T f(t|R_\text{gal}) dt 
\\ 
\dot{\Sigma}_\star(t = 0, R_\text{gal}) &= \Sigma_\star(R_\text{gal}) 
\left[(1 - r) \int_0^T f(t|R_\text{gal})dt\right]^{-1} 
\\ 
&= \Sigma_\star(R_\text{gal} = 0)g(R_\text{gal}) \left[(1 - r) \int_0^T 
f(t|R_\text{gal})\right]^{-1} 
\label{migration:eq:intermediate_1} 
\end{align}\end{subequations} 
where~$(1 - r) \approx 0.6$~is an adequate approximation for a 
\citet{Kroupa2001} IMF~\citep[][see discussion in 
their~\S~2.2]{Weinberg2017b}. 
This expression relates the two unkowns introduced by 
equations~\refp{migration:eq:sfh_terms_of_f} and~\refp{migration:eq:sigma_terms_of_g}. We continue 
by asserting that the integral of the stellar surface density over the area of 
the disc should be equal to the present-day stellar mass of the Milky Way: 
\begin{subequations}\begin{align} 
M_\star^\text{MW} &= \int_0^R \Sigma_\star(R_\text{gal}) 2\pi R_\text{gal} 
dR_\text{gal} 
\\ 
&= \Sigma_\star(R_\text{gal} = 0) \int_0^R g(R_\text{gal}) 2\pi R_\text{gal} 
dR_\text{gal} 
\\ 
\Sigma_\star(R_\text{gal} = 0) &= M_\star^\text{MW} \left[\int_0^R 
g(R_\text{gal}) 2\pi R_\text{gal}dR_\text{gal}\right]^{-1}. 
\label{migration:eq:intermediate_2} 
\end{align}\end{subequations} 
\par 
Now plugging equation~\refp{migration:eq:intermediate_2} into equation 
\refp{migration:eq:intermediate_1}, and then that into equation~\refp{migration:eq:sfh_terms_of_f} 
yields the desired result: 
\begin{subequations}\begin{align} 
&\dot{\Sigma}_\star(t, R_\text{gal}) = Af(t|R_\text{gal})g(R_\text{gal}) 
\\ 
&A = M_\star^\text{MW}\left[(1 - r) \int_0^R g(R_\text{gal})2\pi R_\text{gal} 
dR_\text{gal} \int_0^T f(t|R_\text{gal})dt\right]^{-1}, 
\end{align}\end{subequations} 
where the upper limits should be the maximum radius of star formation and the 
end time of the simulation (15.5 kpc and 13.2 Gyr in this paper). This result 
makes intuitive sense, simply stating that the required normalization of 
$f(t|R_\text{gal})$~is specified by two things: the total stellar mass of the 
Galaxy and how steeply the stellar density falls with increasing radius. As 
long as the assumption that stellar migration does not significantly alter the 
form of~$g(R_\text{gal})$~is not violated, this procedure can be used to 
calculate prefactors for future models of disc galaxies. 

