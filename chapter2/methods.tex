
\section{Methods: The One-Zone Approximation}
\label{bursts:sec:methods}
\subsection{The Gas Supply, Star Formation Rate, and Star Formation Efficiency}
Under the one-zone approximation, the fundamental assumption is instantaneous 
mixing of newly released metals throughout the star-forming gas reservoir. 
In practice, the validity of this approximation depends on the ratio
of the mixing timescale to the depletion timescale, i.e., the average 
time required for an ISM fluid element to be either incorporated into
a star or ejected in an outflow.  For conditions in typical star-forming
disk galaxies, characteristic depletion times are $\sim 0.5-10\text{ Gyr}$
(see discussion of \citet{Weinberg2017b} based on observations
of \citealt{Leroy2008}).  Simulations of turbulent diffusion in disks
imply that azimuthal mixing times are a fraction of an orbital period
while radial mixing times can be much longer, so that ISM mixing will
typically erase azimuthal abundance variations but not radial
gradients \citep{Petit2015,Krumholz2018a}.
In the dwarf galaxy regime, lengthscales are shorter while characteristic
turbulent velocities are comparable, so instantaneous mixing should
be a good approximation azimuthally and may become an adequate 
approximation galaxy-wide.  However, we are unaware of systematic
studies of metal-mixing in the dwarf galaxy regime.
\par
Under the one-zone approximation,
the equations of galactic chemical evolution (GCE)
reduce to a system of integro-differential equations of mass with time, which 
can be integrated numerically. Under this formalism the time-derivative of the 
gas-supply is given by: 
\begin{subequations}\begin{align} 
\label{bursts:eq:mdot_gas} 
\dot{M}_\text{g} &= \dot{M}_\text{in} - \dot{M}_* - \dot{M}_\text{out} + 
\dot{M}_\text{returned} \\ 
&\approx \dot{M}_\text{in} - \dot{M}_*(1 + \eta - r_\text{inst}) \\ 
&= \dot{M}_\text{in} - M_\text{g}\tau_*^{-1}(1 + \eta - r_\text{inst}) 
\end{align}\end{subequations} 
where $\dot{M}_\text{in}$ is the mass infall rate, $\dot{M}_*$ is the SFR, 
$\dot{M}_\text{out}$ is the mass outflow rate, and $\dot{M}_\text{returned}$ 
is the rate of recycling. The star formation efficiency (SFE) timescale is 
defined by $\tau_* = M_\text{g}/\dot{M}_*$, and the parameters $\eta$ and 
$r_\text{inst}$ are discussed further below. 
\texttt{VICE} allows the user to specify the initial gas supply $M_\text{g,0}$ 
and the inflow rate $\dot{M}_\text{in}$ as a function of time, in which case 
the SFR follows from the star formation law $\dot{M}_* = M_\text{g}/\tau_*$. 
Alternatively, the user can specify the star formation history $\dot{M}_*$ 
itself or the gas supply at all times $M_\text{g}(t)$, with the star formation 
law supplying the remaining quantity. In these cases, the infall rate is 
determined implicitly by solving equation~\refp{bursts:eq:mdot_gas}. The former 
approach is somewhat more common in chemical evolution modeling, reflecting 
the expectation that a galaxy's star formation history is ultimately governed 
by the rate at which it accretes gas from the surrounding circumgalactic 
medium. However, a galaxy's star formation history can be estimated 
observationally while its accretion history cannot, and for analytic solution 
it is often more convenient to specify $\dot{M}_*(t)$ rather than 
$\dot{M}_\text{in}(t)$ as shown by~\citet{Weinberg2017b}. For the 
calculations in this paper, we specify $\dot{M}_\text{in}(t)$ and allow the 
SFR to follow from the gas supply unless otherwise specified. 
\par 
As a default value for the SFE timescale we adopt 
$\tau_*$ = 2 Gyr, the typical value found for molecular gas in a wide range of 
star-forming galaxies~\citep{Leroy2008}. The observationally inferred $\tau_*$ 
is lower in some starbursting systems, as short as~$\sim$100 Myr; however, the 
details of this relation are subject to the ongoing debate about the 
CO-to-H$_2$ conversion 
factor~\citep[for details, see the review in][]{Kennicutt2012}. 
Relative to the total gas supply, the SFE timescale will be longer if much of 
the reservoir is in atomic form, roughly 
$\tau_* = (\text{2 Gyr})(1 + M_\text{HI}/M_{\text{H}_2})$. 
\texttt{VICE} allows the 
user to specify $\tau_*$ as a function of time, simultaneously allowing it to 
vary with the gas supply according to the Kennicutt-Schmidt 
relation~\citep{Schmidt1959, Schmidt1963, Kennicutt1998}. If one views the 
gas reservoir as representing an annulus of a disk, with gas surface density 
$\Sigma_\text{g} = M_\text{g}/A_\text{ann}$, then the classic non-linear 
Kennicutt-Schmidt law $\dot{\Sigma}_* \propto \Sigma_\text{g}^{1.5}$ implies 
$\tau_* \propto M_\text{g}^{-0.5}$. We adopt this form in some of our 
calculations below. 

\afterpage{
\clearpage
\begin{landscape}
\begin{table*} 
\caption{Galactic chemical evolution parameters and their fiducial/unperturbed 
values adopted in this paper (if applicable). For further details 
on each parameter, see \texttt{VICE}'s science documentation, available 
at~\url{https://github.com/giganano/VICE/tree/master/docs}. } 
\begin{tabularx}{\linewidth}{l @{\extracolsep{\fill}} l r} 
\hline
\hline
Quantity & Description & Fiducial/Unperturbed Value \\ 
\hline 
$M_\text{g}$ & Gas Supply & $\sim6\times10^9\ M_\odot$ \\ 
$\dot{M}_*$ & Star Formation Rate & $\sim3\ M_\odot\ \text{yr}^{-1}$ \\ 
$\dot{M}_\text{in}$ & Infall Rate & $\sim9\ M_\odot\ \text{yr}^{-1}$ \\ 
$\dot{M}_\text{out}$ & Outflow Rate & 
	$\eta\langle\dot{M}_*\rangle_{\tau_\text{s}}$ \\ 
$\dot{M}_\text{returned}$ & Recycling Rate & 
	Continuous (see equation~\ref{bursts:eq:mdot_returned}) \\ 
$\tau_*$ & Star Formation Efficiency (SFE) Timescale ($M_\text{g}/\dot{M}_*$) & 
	2 Gyr \\ 
$\eta$ & Mass-Loading Factor ($\dot{M}_\text{out} / \dot{M}_*$) & 2.5 \\ 
$\xi_\text{enh}$ & Outflow Enhancement Factor ($Z_\text{out}/Z_\text{ISM}$) & 
	1 \\ 
$\dot{M}_x^\text{CC}$ & Rate of Enrichment from CCSNe & N/A \\ 
$y_x^\text{CC}$ & IMF-integrated fractional yield from CCSNe & 
	O: 0.015; Fe: 0.0012; Sr: $3.5\times10^{-8}$ \\ 
$\dot{M}_x^\text{Ia}$ & Rate of Enrichment from SNe Ia & N/A \\ 
$y_x^\text{Ia}$ & IMF-integrated fractional yield from SNe Ia & 
	O: 0.0; Fe: 0.0017; Sr: 0.0 \\ 
$\dot{M}_x^\text{AGB}$ & Rate of Enrichment from AGB stars & N/A \\ 
$y_x^\text{AGB}(m_\text{to} | Z)$ & Fraction yield from an AGB star of mass 
$m_\text{to}$ and metallicity $Z$ & \citet{Cristallo2011} \\ 
$r(t)$ & Cumulative Return Fraction (CRF) & N/A \\ 
$h(t)$ & Main Sequence Mass Fraction (MSMF) & N/A \\ 
$Z_\text{ISM}$ & Total Metallicity by Mass of the ISM & N/A \\ 
\hline
\end{tabularx}
\label{bursts:tab:docs} 
\end{table*}
\end{landscape}
\clearpage
}

\subsection{The Cumulative Return Fraction} 
\label{bursts:sec:crf} 
The cumulative return fraction (CRF) $r(t)$ is the fraction of a stellar 
population's mass formed at $t$ = 0 that has been returned to the ISM by a 
time $t$ through stellar winds or supernova explosions. In \texttt{VICE}, we 
calculate $r(t)$ approximately by assuming that all stars with initial mass 
$M > 8 M_\odot$ leave a remnant of 1.44 $M_\odot$ while those less than 8 
$M_\odot$ leave remnants of mass $0.394 M_\odot + 0.109 M$~\citep{Kalirai2008}. 
In these calculations, the main sequence turnoff mass at a time $t$ following 
the formation of a stellar population is assumed to be $M_\text{to}/M_\odot 
\approx (t/\text{10 Gyr})^{-1/3.5}$, the same form as adopted 
in~\citet{Weinberg2017b}. While this formula is less accurate for high 
$M_\text{to}$, the return timescale for these stars is much shorter than 
other chemical evolution timescales anyway, so the approximation is adequate. 
\par
\texttt{VICE} calculates the time-dependent return rate from all previous 
stellar generations as: 
\begin{equation}
\label{bursts:eq:mdot_returned} 
\dot{M}_\text{returned}(t) = \int_0^t\dot{M}_*(t - t')\dot{r}(t')dt' . 
\end{equation} 
Alternatively, one can make the approximation that all mass (from AGB stars as 
well as CCSNe) is returned instantaneously, in which case: 
\begin{equation} 
\label{bursts:eq:mdot_returned_inst}
\dot{M}_\text{returned} = r_\text{inst}\dot{M}_* . 
\end{equation}
For a Kroupa IMF, the CRF is $r(t) \approx$ 0.37, 0.40, and 0.45 after 1, 2, 
and 10 Gyr, and~\citet{Weinberg2017b} shows that the difference between 
[$\alpha$/Fe]-[Fe/H] evolution with the time-dependent return of 
equation~\refp{bursts:eq:mdot_returned} and the instantaneous approximation with 
$r_\text{inst}$ = 0.4 is very small. Nonetheless, numerical implementation of 
equation~\refp{bursts:eq:mdot_returned} is neither difficult nor time-consuming, so we 
use continuous recycling throughout this paper. We note that 
equation~\refp{bursts:eq:mdot_returned_inst} is not equivalent to the ``instantaneous 
recycling approximation'' as that term is most frequently used, where it 
implies instantaneous return of \textit{newly produced} elements as well as the 
mass and metals that stars are born with. The full instantaneous recycling 
approximation is accurate for pure-CCSN elements if the star formation 
history is smooth on timescales of~$\sim$100 Myr, but it is not an accurate 
description of SN Ia enrichment. 
While our one-zone models assume instantaneous mixing,
they {\it do not} assume instantaneous recycling for enrichment 
by SN Ia or AGB stars.

\subsection{The Mass Loading Factor} 
For the outflow mass loading factor $\eta$ we adopt a default value of 2.5, 
the same as~\citet{Weinberg2017b}, with the result that our models 
approach approximately solar abundances at late times given our adopted CCSN 
and SN Ia yields. However, as noted in \S~\ref{bursts:sec:intro}, we also 
consider the possibility that the outflow rate is not tied to the 
instantaneous SFR but to some time-averaged value. This introduces an 
additional parameter, the smoothing timescale $\tau_\text{s}$, defined such 
that 
\begin{equation}\begin{split}
\label{bursts:eq:tau_s}
\dot{M}_\text{out} &= \eta\langle\dot{M}_*\rangle_{\tau_\text{s}} \\ 
&= \begin{dcases}
\frac{\eta}{\tau_\text{s}}\int_{t - \tau_\text{s}}^{t} \dot{M}_*(t') dt' & 
(t > \tau_\text{s}) \\ 
\frac{\eta}{t}\int_{0}^{t} \dot{M}_*(t') dt' & (0 \leq t \leq \tau_\text{s}) . 
\end{dcases}
\end{split}\end{equation} 
If galactic winds are driven primarily by massive star winds, radiation 
pressure, and CCSNe, then the effective smoothing timescale is likely to be 
short ($\tau_\text{s}\sim$ 50 Myr), and smoothing will have little impact on 
chemical evolution if the SFR is smooth on these timescales. However, if SNe 
Ia play a central role in driving winds, then effective smoothing times as 
long as $\tau_\text{s}\sim$ 1 Gyr are possible, altering the relative ejection 
of CCSNe and SNe Ia elements from a shorter duration starburst. Cosmic ray 
feedback could also produce an intermediate smoothing time, because the 
energy deposited by CCSNe can be temporarily stored in cosmic rays before 
building up sufficiently to drive a wind. While \texttt{VICE} allows the user 
to specify $\eta$ as a function of time, we do not consider models with a 
time-varying $\eta$ here. 

\subsection{CCSNe} 
\label{bursts:sec:ccsne} 
Following~\citet{Weinberg2017b}, we implement in \texttt{VICE} the 
instantaneous explosion approximation to CCSNe. This is a good approximation, 
because the lifetimes of CCSN progenitors ($\lesssim$40 Myr for the least 
massive ones) are much shorter than the relevant timescales of 
GCE. In our models, a given yield of some element X is ejected 
simultaneously with the formation of a stellar population at all timesteps: 
\begin{equation} 
\label{bursts:eq:mdot_ccsne}
\dot{M}_\text{x}^\text{CC} = y_\text{x}^\text{CC}(Z)\dot{M}_* ~,
\end{equation} 
where $y_\text{x}^\text{CC}$ is the fraction of the stellar population's 
\textit{total} mass that is converted to the element $x$ at a metallicity 
$Z$.
The CCSN yield is
\begin{equation} 
\label{bursts:eq:frac_yield}
y_x^\text{CC} = \ddfrac{
	\int_{m_{\rm SN}}^u m_x \frac{dN}{dm}dm 
}{ 
	\int_l^u m \frac{dN}{dm}dm 
}
\end{equation} 
where $m_x$ is the mass of the element $x$ ejected in the explosion of a star 
of initial mass $m$, and $dN/dm$ is the assumed stellar IMF, for which we adopt 
the~\citet{Kroupa2001} form throughout this work. We also adopt $l$ = 0.08 
$M_\odot$ and $u$ = 100 $M_\odot$ as the lower and upper mass limits of the 
IMF and $m_{\rm SN} = 8\ M_\odot$ 
as the minimum progenitor mass for a CCSN explosion. 
If some stars above $m_{\rm SN}$ implode to black holes instead of 
exploding as supernovae, they will have much lower (possibly zero)
values of $m_X$ (e.g., \citealt{Sukhbold2016}).
\par 
Upon request~\texttt{VICE} will calculate $y_\text{x}^\text{CC}$ for a given 
element and metallicity from literature tables.
It also allows users to adopt any numerical value or
user-constructed functions of $Z$ to describe the yield for any element in 
its simulations. 
In practice, supernova nucleosynthesis studies determine the value of 
$m_\text{x}$ for of order 10 values of $m$ at a specified metallicity and 
rotational velocity. To compute the numerator of equation~\refp{bursts:eq:frac_yield}, 
\texttt{VICE} linearly interpolates $m_\text{x}$\footnote{
	linearly in $m$, not $\log m$
} values between the two surrounding $m$ values in the available yield 
grid, or linearly extrapolates $m_\text{x}$ values from the two highest $m$ 
values in the grid if it does not extend to 100 $M_\odot$. 
\par 
We discuss our adopted O and Fe CCSN yields 
in~\S~\ref{bursts:sec:yields}. 
We base our choices on \citet{Weinberg2017b}, but \citet{Weinberg2017b}
did not investigate Sr enrichment, which we 
are interested in as a tracer of s-process nucleosynthesis in both CCSN and 
AGB stars. For this reason, we conduct a thorough investigation of 
CCSN Sr yields and the metallicity dependence thereof. We reserve this 
discussion to~\S~\ref{bursts:sec:sr_nuc}, which focuses on Sr nucleosynthesis. 

\subsection{The SN Ia Delay-Time Distribution (DTD)} 
\label{bursts:sec:dtd} 
We define $R_\text{Ia}(t)$ to be the rate of SNe Ia per unit stellar mass 
formed at a time $t$ following an episode of star formation. 
Following~\citet{Weinberg2017b} (see Appendix A therein), we set: 
\begin{equation} 
\label{bursts:eq:mdot_ia} 
M_\text{x}^\text{Ia} = y_\text{x}^\text{Ia}\langle\dot{M}_*\rangle_\text{Ia} 
\end{equation} 
where 
\begin{equation} 
\label{bursts:eq:y_x_Ia} 
y_\text{x}^\text{Ia} \equiv m_\text{x}^\text{Ia} 
\int_{t_\text{D}}^{t_\text{max}} R_\text{Ia}(t)dt = 
m_\text{x}^\text{Ia}\frac{N_\text{Ia}}{M_*} 
\end{equation} 
is the fractional yield of element X from all SNe Ia that would explode 
between the minimum delay time $t_\text{D}$ and a specified maximum time 
$t_\text{max}$. Here $m_\text{x}^\text{Ia}$ is the average mass yield of the 
element X per SN Ia, $M_*$ is the mass of the stellar population, and 
\begin{equation}
\label{bursts:eq:mdotstarIa} 
\langle\dot{M}_*\rangle_\text{Ia} \equiv \ddfrac{
	\int_0^t \dot{M}_*(t - t')R_\text{Ia}(t')dt' 
}{
	\int_{t_\text{D}}^{t_\text{max}} R_\text{Ia}(t')dt' 
} 
\end{equation} 
is the time-averaged SFR weighted by the SNe Ia DTD. In implementation, 
\texttt{VICE} enforces $t_\text{max}$ = 15 Gyr always, though provided one is 
consistent in equations~\refp{bursts:eq:y_x_Ia} and~\refp{bursts:eq:mdotstarIa}, the result 
of equation~\refp{bursts:eq:mdot_ia} is independent of the choice of $t_\text{max}$. 
This formulation implicitly assumes that $R_\text{Ia}$ and 
$m_\text{x}^\text{Ia}$ are independent of the birth population's metallicity. 
As discussed further in \S~\ref{bursts:sec:yields} below, we adopt a power-law DTD 
with $R_\text{Ia} \propto t^{-1.1}$ and a minimum time delay of 
$t_\text{D}$ = 150 Myr. \texttt{VICE} allows the user to specify alternative 
forms for the DTD, including user-constructed functional forms. 

\subsection{AGB Stars} 
For AGB enrichment, we implement in \texttt{VICE} an algorithm that tracks the 
mass rate of change of a single stellar population to determine the mass in 
dying stars at each timestep. The rate of mass enrichment of an element $x$ 
from AGB stars is then given by 
\begin{equation} 
\label{bursts:eq:mdot_agb} 
\dot{M}_\text{x}^\text{AGB} = -\int_0^t y_\text{x}^\text{AGB}(m_\text{to}
(t - t'), Z_\text{ISM}(t'))\dot{M}_*(t')\dot{h}(t - t')dt' 
\end{equation} 
where $y_\text{x}^\text{AGB}$ is the yield of a star of mass $m$ and 
total metallicity $Z$, and $h$ is the main sequence mass fraction, 
defined to be the fraction of a stellar population's mass that is in the form 
of main sequence stars at a time $t$ following its formation. By definition, 
$h$ = 1 at $t$ = 0, and declines monotonically; hence the minus sign in 
equation~\refp{bursts:eq:mdot_agb}. $h$ is fully described by the adopted 
stellar IMF and the mass-lifetime relation (see \texttt{VICE}'s Science 
Documentation for further details). 

\subsection{Adopted Nucleosynthetic Yields} 
\label{bursts:sec:yields} 
For CCSN yields of O and Fe, we adopt the same values 
as~\citet{Weinberg2017b}, $y_\text{O}^\text{CC}$ = 0.015 and 
$y_\text{Fe}^\text{CC}$ = 0.0012, independent of metallicity. The former is 
approximately the value computed from the yields of~\citet{Chieffi2004} 
for solar metallicity stars assuming a Kroupa IMF in which all stars with M = 
8 - 100 $M_\odot$ explode. CCSN iron yields are difficult to predict from 
first principles; our choice yields a plateau at [O/Fe] $\approx$ +0.45, in 
reasonable agreement with observations. Although we investigate Sr as a 
representative example of an AGB element, it is also expected to have a CCSN 
contribution. In \S~\ref{bursts:sec:sr} we examine the impact of various assumptions 
of the form of $y_\text{Sr}^\text{CC}$, including one with no 
metallicity dependence, one that depends linearly on $Z$, another with a 
$y_\text{Sr}^\text{CC} \propto 1 - e^{-kZ}$ dependence, and one in which 
$y_\text{Sr}^\text{CC} = 0$ as a limiting case describing pure AGB enrichment. 
\par
For the SN Ia iron yield we adopt $y_\text{Fe}^\text{Ia}$ = 0.0017, similar to 
the values used by~\citet{Schoenrich2009a},~\citet{Andrews2017}, 
and~\citet{Weinberg2017b}. This value is based on a normalization of the 
SNe Ia DTD that yields $N_\text{Ia}/M_* = \int_{t_\text{D}}^{t_\text{max}}
R_\text{Ia}(t)dt = 2.2\times10^{-3} M_\odot^{-1}$, consistent with 
$(2\pm1)\times10^{-3} M_\odot^{-1}$ from~\citet{Maoz2012a}, and 
$m_\text{Fe}^\text{Ia}$ = 0.78 $M_\odot$ from the W70 explosion model 
of~\citet{Iwamoto1999}. Because this enrichment channel is negligible for 
O and Sr, we adopt $y_\text{Sr}^\text{Ia}$ and $y_\text{O}^\text{Ia}$ = 0 
throughout this work. As noted in \S~\ref{bursts:sec:dtd}, we adopt a 
$t^{-1.1}$ power-law DTD, again motivated by~\citet{Maoz2012a}, with a minimum 
delay time of $t_\text{D}$ = 150 Myr. In principle, $t_\text{D}$ could be as 
short as the lifetime of the most massive stars that produce white dwarfs 
(roughly 40 Myr), but it is not clear empirically whether the t$^{-1.1}$ 
power-law extends to such small $t$. As a rule of thumb, it is useful to 
remember that a $t^{-1}$ power-law DTD would yield equal numbers of SNe Ia 
per logarithmic time interval (i.e. the same number between 0.1 - 1 Gyr and 
1 - 10 Gyr). Thus 1 Gyr is the approximate characteristic time for half of the 
SN Ia iron to be produced. If $t_\text{D}$ is as short as 0.05 Gyr, then 
about 20\% of SNe Ia explode between 0.05 and 0.15 Gyr, enough to noticably 
shift the ``knee'' of the [$\alpha$/Fe]-[Fe/H] tracks. For our default 
$t_\text{D}$ = 0.15 Gyr, these tracks are nearly identical to those of an 
exponential DTD with the same
normalization~\citep[see figure 11 of ][]{Weinberg2017b}.
\par 
Recently~\citet{Maoz2017} argued for a lower DTD normalization of 
$N_\text{Ia}/M_* = (1.3\pm0.1)\times10^{-3} M_\odot^{-1}$ for a Kroupa IMF, 
based on comparisons of the cosmic star formation history and the 
redshift-dependent SN Ia rate derived from cosmological surveys. Adopting this 
lower normalization would require us to adopt lower values of 
$y_\text{O}^\text{CC}$, $y_\text{Fe}^\text{CC}$, and $\eta$ to reproduce the 
observed [O/Fe]-[Fe/H] tracks in the Milky Way, reducing each by roughly a 
factor of two. Such a change is physically plausible, because many of the 
high-mass stars with the highest oxygen yields may collapse to black holes 
instead of exploding as CCSNe (see discussion by, e.g.,
~\citealp{Pejcha2015},~\citealp{Sukhbold2016}, and observational evidence 
of~\citealp{Gerke2015},~\citealp{Adams2017}). These changes would not alter our 
qualitative conclusions below, but they would change the detailed form of 
evolutionary tracks and element ratio distributions. \citet{Brown2019} found 
that the local specific SN Ia rate scales strongly (and inversely) with 
galaxy stellar mass, and they argue that this dependence may imply a 
metallicity-dependent $R_\text{Ia}(t)$ in addition to a DTD that produces more 
SNe Ia at early times. Adopting a metallicity-dependent $y_\text{Fe}^\text{Ia}$ 
would have a larger qualitative impact on our models (though as a practical 
matter it would be straight-forward to implement within \texttt{VICE}). We 
reserve a more thorough investigation of empirical constraints on elemental 
yields to future work. 
\par 
The AGB yields of s-process elements depend strongly on both stellar mass and
birth metallicity. It is therefore not feasible to specify single yield values 
or simple time-dependent functional forms. Instead, \texttt{VICE} implements 
a grid of fractional yields on a table of stellar mass and metallicity. At 
each timestep, and for each element, it then determines the appropriate yield 
$y_\text{x}^\text{AGB}$ in equation~\refp{bursts:eq:mdot_agb} via bilinear 
interpolation between elements on the grid. The current version of 
\texttt{VICE} allows users to adopt either the~\citet{Cristallo2011} 
or~\citet{Karakas2010} yield tables, and we adopt the former for calculations 
in this paper. A future version of \texttt{VICE} will likely include more yield 
tables as well as the capability to handle user-specifications on the AGB 
yields of each element. We provide further discussion of Sr yields in 
\S~\ref{bursts:sec:sr}. 

\subsection{Illustrations}
For smooth star formation histories, \texttt{VICE} yields
[O/Fe]-[Fe/H] tracks similar to those of
\citet{Andrews2017}, \citet{Weinberg2017b}, 
and \citet{Freudenburg2017}, who present comparisons to data
for Milky Way stellar populations.  In this paper we focus on the ways
that starbursts and subtler perturbations to the star formation
history influence abundance tracks and distributions, and we do not
attempt to model or interpret current observational data.  For illustrative
purposes, we present in Appendix A a comparison of \texttt{VICE} 
predictions to the dwarf galaxy abundance data of 
\cite{Kirby2010}, for which low characteristic metallicities
require quite different parameter choices from the Milky Way.
