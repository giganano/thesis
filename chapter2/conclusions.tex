
\section{Conclusions}
\label{bursts:sec:conclusion} 

We have studied one-zone chemical evolution models tracking the enrichment of 
oxygen, iron, and strontium with the goal of understanding the impact of star 
formation bursts on elemental abundance ratios. To this end, we have developed 
the \texttt{Versatile Integrator for Chemical Evolution} (\texttt{VICE}), a 
\texttt{python} package optimized for handling highly non-linear chemical 
evolution models. With this new tool, we first simulated gas-driven 
starbursts, whereby an amount of gas comparable to the current ISM mass of a 
galaxy is added to the ISM on timescales shorter than the depletion time. 
These starburst models predict hooks in the [O/Fe]-[Fe/H] plane; the rapid 
addition of pristine gas first causes a reduction in [Fe/H] at fixed [O/Fe], 
then the elevated rate of CCSNe relative to SNe Ia drives the ISM to higher 
[O/Fe] and [Fe/H], and finally the onset of SNe Ia associated with the 
starburst pushes the ISM back toward the [O/Fe]-[Fe/H] track of the unperturbed 
(i.e. no-burst) model. The rate at which extra gas is added to the galaxy 
affects the detailed shape of these jump-and-hook trajectories. 
Although this paper focuses on characterizing model predictions
rather than interpreting data, we provide an illustrative comparison to
Kirby et al.'s (\citeyear{Kirby2010}) abundance measurements for 
Milky Way dwarf satellites in Appendix~A.
\par 
Our unperturbed constant-SFR models predict one peak in the [O/Fe] distribution 
at the ``plateau'' ratio of CCSN O and Fe yields, and a second peak associated 
with the late-time equilibrium in which CCSN and SN Ia rates are equal. Our 
gas-driven starburst models predict a third peak in the [O/Fe] distribution, 
associated with stars that form out of the $\alpha$-enhanced ISM 
during/following the burst before SNe Ia have driven evolution back toward the 
unperturbed evolutionary track. The peak is centred near the value of [O/Fe] 
at the top of the hook in the [O/Fe]-[Fe/H] trajectory, and its location and 
shape are insensitive to the timescale on which the gas is added, provided 
that this timescale is short compared to the depletion time. Earlier 
starbursts produce this third peak at higher [O/Fe] because they arise when 
the starting value of [O/Fe] is further from its eventual equilibrium. Thus, 
even without accurate ages for individual stars, the existence of extra 
peaks in the [O/Fe] distribution (or [X/Fe] distribution for other 
$\alpha$-elements) can provide an observable diagnostic for past bursts of 
galactic star formation, and the locations of these peaks can provide 
estimates of the timing of these bursts. 
\par 
A gas-driven starburst could arise from the merger of a gas rich system or a 
temporary increase in accretion rate. A starburst can also arise from a 
temporary increase in star formation efficiency, consuming the available gas 
more quickly, perhaps because of a dynamical disturbance that does not 
increase the gas supply. The evolutionary tracks of efficiency-driven 
starbursts differ in form from those of gas-driven starbursts, first because 
there is no drop in [Fe/H] before the increase in [O/Fe], and second because 
[O/Fe] loops below the track of the unperturbed model once the post-burst 
gas supply is depleted, which allows the CCSN rate to fall well below the rate 
of SNe Ia from stars that formed during the burst. With sufficiently precise 
data, the [O/Fe] distribution of an efficiency-driven burst can be 
distinguished from that of a gas-driven burst, in part by the shape of the 
[O/Fe] peak for $\alpha$-enhanced stars formed during the burst, and in part 
by the presence of an additional population of $\alpha$-deficient stars. 
\par 
In short, the chemical response to a simple starburst is driven by a 
perturbation of the CCSN and SN Ia rates. Conventionally, one-zone models tie 
the outflowing wind to the instantaneous star formation rate (i.e. 
$\dot{M}_\text{out} = \eta\dot{M}_*$), which in turn ties it to the CCSN rate. 
If SNe Ia contribute to the outflowing wind, then a better approximation of 
the outflow rate would be one that is tied to a time-averaged SFR (i.e. 
$\dot{M}_\text{out} = \eta\langle\dot{M}_*\rangle_{\tau_\text{s}}$), where 
$\tau_\text{s}$ is the outflow smoothing time. Nonzero $\tau_\text{s}$ allows 
the ISM to retain more gas at the onset of a starburst, because the outflow is 
more sensitive to the preburst SFR. The ISM is then gas-poor in the decay of 
the starburst, because the outflow is most sensitive to the elevated SFR from 
the recent burst. Varying $\tau_\text{s}$ between 0 and 1 Gyr has minimal 
impact on the predicted [O/Fe]-[Fe/H] trajectory or the stellar [O/Fe] 
distribution of gas-driven starbursts. However, efficiency-driven burst models 
with $\tau_\text{s}$ = 0.5 - 1 Gyr exhibit wider loops and produce lower 
[$\alpha$/Fe] stars than in the $\tau_\text{s}$ = 0 model. These 
$\alpha$-deficient stars are produced late in the burst when the ISM is 
gas-poor, an effect that is magnified by a nonzero $\tau_\text{s}$ because 
the gas outflow rate is higher and the CCSN rate is lower. 
\par 
While our simplest starburst scenarios are either gas- or efficiency-driven, 
there is observational evidence for starbursts driven by both 
an increase in the gas supply and an increase in the 
efficiency~\citep[][and the citations therein]{Kennicutt2012}. As a simple 
example of a ``hybrid'' starburst, we considered a model with a rapid influx 
of gas and an SFE timescale $\tau_* \propto M_\text{g}^{-1/2}$ as suggested 
by the Kennicutt-Schmidt law~\citep{Schmidt1959, Schmidt1963, Kennicutt1998}. 
For $\tau_\text{s}$ = 0 or 0.5 Gyr, the [O/Fe]-[Fe/H] tracks of this model are 
nearly the same as those of the gas-driven constant-$\tau_*$ model. For 
$\tau_\text{s}$ = 1 Gyr, the hybrid model shows aspects of both gas-driven and 
efficiency-driven models, including a population of $\alpha$-deficient stars. 
\par 
The AGB models of~\citet{Cristallo2011} predict Sr yields that are strongly 
dependent on metallicity and dominated by 2 - 4 $M_\odot$ stars. Predicted 
CCSN yields of Sr are sensitive to rotationally induced mixing; 
the~\citet{Limongi2018} yields for non-rotating progenitors vs. progenitors 
with $v_\text{rot}$ = 150 km s$^{-1}$ differ by 1-2 orders of magnitude, with 
strong but differing metallicity dependence. Near solar metallicity, the AGB 
yields and non-rotating CCSN yields are comparably important, but the 
$v_\text{rot}$ = 150 km s$^{-1}$ CCSN yields would outweight AGB yields by a 
large factor. 
\par 
Reproducing the approximately flat trend of [Sr/Fe] vs. [Fe/H] found in the 
Milky Way and in dwarf satellites~\citep{Mishenina2019, Hirai2019} requires 
an additional source of Sr with a yield that is nearly independent of 
metallicity, perhaps the neutrino-driven winds from newly formed neutron 
stars~\citep{Thompson2001, Vlasov2017, Thompson2018}. For any of these CCSN 
yield models, the tracks of [Sr/Fe] vs. [Fe/H] in starburst models are complex, 
affected by the yield metallicity dependence and by the differing timescales 
of CCSN, AGB, and SN Ia enrichment. Tracks of [Sr/O] vs. [O/H] are simpler 
because they are independent of SN Ia enrichment. However, the total range 
of [Sr/Fe] or [Sr/O] induced by starbursts is small, typically 0.05-0.1 dex, 
and in combination with yield uncertainties this small dynamic range makes it 
difficult to use Sr abundances as a diagnostic of starburst behavior. 
\par 
In addition to strong (factor of~$\sim$2) starbursts, we have investigated 
models with 10-20\% sinusoidal modulations of a constant SFR, induced by 
variations in infall rate or star-formation efficiency. These models predict 
[O/Fe]-[Fe/H] tracks that oscillate about the prediction of the constant SFR 
model. The produce a multi-peaked structure in [O/Fe] distributions, though 
in the presence of observational errors these peaks would likely merge into 
a broader distribution. These variations do not produce a bimodal [O/Fe] 
distribution, so they are not the origin of the observed separation of thin 
and thick disk sequences~\citep[e.g.][]{Bensby2003, Hayden2015, 
BertranDeLis2016}. However, moderate variations in SFR could be a source of 
scatter in [O/Fe] along these sequences. With our adopted parameter values, 
our smooth evolution models approximately reproduce the observed high-$\alpha$ 
sequence. SFR oscillations produce a spread of~$\sim$0.05-0.1 dex in [O/Fe] 
for [Fe/H] $\gtrsim$ -0.4, but they cannot produce scatter near the 
high-$\alpha$ plateau of this sequence because the enrichment of those stars 
is dominated by CCSN in any case. 
\par 
Motivated by findings on the recent star formation history of the 
Milky Way by~\citet{Mor2019} and~\citet{Isern2019}, we explored 
models that exhibit slow, factor of~$\sim$2 increases in the SFR at lookback 
times of~$\sim$2 Gyr, adopting a simple gaussian with dispersion of $\sigma$ = 
1 Gyr to describe the starburst. A late-time, slow starburst may help to 
explain otherwise puzzling features of the age-abundance relations observed 
in APOGEE~\citep{Martig2016,SilvaAguirre2018,Feuillet2018,Feuillet2019}, such 
as young stars with mild $\alpha$-enhancements and young median ages of solar 
metallicity or $\alpha$-deficient stars. Complete modeling of these 
observables requires multizone models that account for radial mixing of 
stellar populations, and we reserve such investigations to future work. 
\par 
Throughout this paper we have adopted an O yield similar to those predicted 
by~\citet{Chieffi2004} and~\citet{Chieffi2013}, assuming a Kroupa IMF in which 
all stars with $M > 8\ M_\odot$ explode. With this yield, evolving to solar 
metallicity requires fairly strong outflows, with $\eta$ = 2.5
\citep[e.g.,][]{Finlator2008, Peeples2011, Andrews2017, Weinberg2017b}.
% (e.g.,~\citealp{Finlator2008};~\citealp{Peeples2011};~\citetalias{Andrews2017}; 
% \citetalias{Weinberg2017}).
With lower IMF-averaged SN yields, which could 
arise if many massive stars form black holes instead of exploding, lower 
values of $\eta$ would be needed to reach the same final metallicity.  
Results for lower yield, lower $\eta$ models would differ in detail from those 
presented here, mainly because the depletion time $\tau_\text{dep} = 
\tau_*/(1 + \eta - r_\text{inst})$ would be longer for the same $\tau_*$. 
However, we
have investigated several of our models in which both yields and $\eta$ are 
reduced by a factor of~$\sim$2 and found that our qualitative conclusions 
still hold. 
\par 
We have released \texttt{VICE} as open-source software under the MIT 
license. Source code, installation instructions, and documentation can be 
found at~\url{http://github.com/giganano/VICE.git}. We also include code that 
runs the simulations of our models and produces the figures in this paper. 
