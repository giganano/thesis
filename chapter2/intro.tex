
\begin{center}
\textit{
	The results presented in this chapter are part of a first-author paper
	published in Monthly Notices of the Royal Astronomical Society, in which I
	led the majority of the analysis and writing.
}
\end{center}

\section{Introduction}
\label{bursts:sec:intro}
The elemental abundances and abundance ratios of stars encode information 
about the history of galactic enrichment and about the stellar processes that 
produce the elements. The ratio of $\alpha$-element abundances to the iron 
abundance is an especially important diagnostic, because the $\alpha$-elements 
(e.g. O, Mg, and Si) are produced primarily by massive stars with short 
lifetimes, while Fe is also produced in substantial amounts by Type Ia 
supernovae (SNe Ia) that explode after a wide range of delay times. In simple 
chemical evolution models with smooth star formation histories, the track of 
[$\alpha$/Fe] vs. [Fe/H]\footnote{
	We follow conventional notation in which [X/Y] $\equiv\ \log_{10}(X/Y) 
	- \log_{10}(X_\odot/Y_\odot)$.  
} first follows a plateau that reflects the IMF\footnote{
	IMF: Initial Mass Function 
}-averaged yield of core collapse supernovae (CCSNe), then turns downward as 
SNe Ia begin to add Fe without associated $\alpha$-elements. If the model has 
continuing gas accretion, then the [Fe/H] and [$\alpha$/Fe] ratios tend to 
approach an equilibrium in which the production of new elements is balanced by 
dilution from freshly accreted gas and by depletion of metals from new star 
formation or outflows~\citep{Larson1972, Finlator2008, Andrews2017,
Weinberg2017b}.
% (\citealp[][hereafter 
% \citetalias{Andrews2017}]{Larson1972, Finlator2008, Andrews2017}; 
% \citealp[][hereafter \citetalias{Weinberg2017}]{Weinberg2017}). 
\par\null\par\null\par 
In this paper we examine the impact of starbursts - sudden and temporary 
increases in the star formation rate - which perturb chemical evolution by 
temporarily boosting the rate of CCSNe relative to SNe Ia from earlier 
generations of stars. We adopt one-zone evolution models in which stars form 
from and enrich a fully mixed gas reservoir subject to accretion and outflow
(see, e.g. \citealp{Schmidt1959, Schmidt1963, Larson1972, Tinsley1980} for 
classical examples; \citealp{Weinberg2017b, Andrews2017} for 
more recent work). Although idealized, one-zone models may be a reasonable 
approximation for the evolution of dwarf galaxies. The Milky Way can be 
modeled as an annular sequence of one-zone models ~\citep{Matteucci1989}, 
which may be connected by processes such as the radial migration of 
stars~\citep{Schoenrich2009a, Minchev2017} and radial gas 
flows~\citep{Lacey1985, Bilitewski2012}. 
\par 
The [$\alpha$/Fe]-[Fe/H] tracks observed in the inner Milky Way agree well with 
the predictions of a one-zone model in which the star-forming gas disk 
contracts vertically over a period of~$\sim$2 Gyr~\citep{Hayden2015, 
Freudenburg2017}. In the solar neighborhood, stars with high and low vertical 
velocities trace distinct [$\alpha$/Fe]-[Fe/H] tracks known as the 
``high-$\alpha$'' and ``low-$\alpha$'' sequences~\citep{Bensby2003, 
Hayden2015}, a bimodality whose origin is still not fully understood. 
\citet{Andrews2017} and~\citet{Weinberg2017b} systematically 
investigate the influence of model parameters on the [$\alpha$/Fe]-[Fe/H] 
tracks of one-zone models with smooth star formation histories, with 
particlular attention to the role of outflows in regulating the equilibrium 
metallicity. In agreement with previous studies of the galaxy mass-metallicity 
relation~\citep[e.g. ][]{Dalcanton2007, Finlator2008, Peeples2011, Zahid2012}, 
they find that achieving a solar metallicity interstellar medium (ISM) requires 
strong outflows, with a mass-loading factor $\eta\equiv\dot{M}_\text{out}/
\dot{M}_*\approx2.5$ for a~\citet{Kroupa2001} IMF where every star of mass 
8 - 100 $M_\odot$ explodes as a CCSN with the yields predicted 
by~\citet{Chieffi2004,Chieffi2013}. 
\par 
\citet{Gilmore1991} investigated the impact of a bursty star formation history 
on [O/Fe]-[Fe/H] tracks, focusing on application to the Large Magellanic 
Cloud. \citet{Weinberg2017b} investigated a model in which a sudden change 
of star formation efficiency induces a starburst, causing an upward jump in 
[O/Fe] followed by a return to equilibrium (see their Figure 9). In this 
paper we study the impact of starbursts more systematically, showing the 
different forms of [O/Fe]-[Fe/H] tracks and stellar [O/Fe] distributions for 
bursts induced by a sudden influx of gas, a boost in gas accretion rate, or 
an increase of star formation efficiency. We also investigate the 
connection between starbursts and winds, considering the possibility that 
outflows are tied to a time-averaged star formation rate (SFR) instead of the 
instantaneous SFR that governs the rate of CCSN enrichment. In addition to 
O and Fe, we examine strontium (Sr) as a representative element that has both 
a CCSN contribution and an asymptotic giant branch (AGB) star contribution 
with a metallicity dependent yield. 
\par 
To this end we have developed a publicly available\footnote{
	\url{https://github.com/giganano/VICE.git}
} \texttt{python} package, the \texttt{Versatile Integrator for Chemical 
Evolution} (\texttt{VICE}), which solves the integro-differential equations of 
a one-zone chemical evolution model. 
Compared to \texttt{flexCE}~\citep{Andrews2017}\footnote{
	\url{https://github.com/bretthandrews/flexce}
}, \texttt{VICE} has a simpler methodology in that it works directly from 
user-specified IMF-averaged yields rather than drawing CCSNe stochastically 
from the IMF of massive stars. \texttt{VICE} focuses instead on versatility 
in specifying star formation histories, gas accretion histories, and star 
formation laws as arbitrary functions of time, and it will automatically 
compute yield tables from a variety of sources if requested~\citep[e.g.][among 
others to be added in subsequent versions]{Woosley1995, Iwamoto1999, 
Chieffi2004, Karakas2010, Cristallo2011, Seitenzahl2013, Limongi2018}. With a 
backend written in \texttt{C}, \texttt{VICE} also achieves powerful computing 
speeds while maintaining this level of flexibility. We anticipate adding 
further capabilities to \texttt{VICE} in the future, including extensions to 
multizone models.
\par 
Our models in this paper are motivated primarily by considerations of dwarf 
galaxies, which often show evidence of bursty star formation 
histories~\citep[e.g.][]{Weisz2011, Weisz2014a}. However, even local variations 
in star formation induced by passage of gas through a spiral arm can induce 
some of these effects, damped mainly by the fact that such events typically 
convert only a small fraction of the available gas into 
stars~\citep{Weinberg2017b}. In their hydrodynamic simulations of disk 
galaxy formation,~\citet{Clarke2019} find that massive clumps in young 
gas-rich disks convert much of their gas into stars and therefore self-enrich, 
following tracks in [$\alpha$/Fe]-[Fe/H] space that resemble those of our 
efficiency-induced starburst models below. They propose that a superposition 
of such bursts is responsible for the high-$\alpha$ sequence observed in the 
Milky Way [$\alpha$/Fe]-[Fe/H] diagram. In addition to bursts, we investigate 
here the effect of slow sinusoidal variations in SFR, finding that these less 
dramatic variations could produce scatter in [$\alpha$/Fe] at fixed [Fe/H], at 
least along the low-$\alpha$ sequence. 

