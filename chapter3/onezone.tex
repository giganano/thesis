
\section{Galactic Chemical Evolution}
\label{dga:sec:onezone}

One-zone GCE models connect the star formation and accretion histories of
galaxies to the enrichment rates in the ISM through prescriptions for
nucleosynthetic yields, outflows, and star formation efficiency (SFE) within
a simple mathematical framework.
Their fundamental assumption is that newly produced metals mix instantaneously
throughout the star-forming gas reservoir.
In detail, this assumption is valid as long as the mixing timescale is short
compared to the depletion timescale (i.e., the average time a fluid element
remains in the ISM before getting incorporated into new stars or ejected in an
outflow).
Based on the observations of~\citet{Leroy2008},~\citet{Weinberg2017b} calculate
that characteristic depletion times can range from~$\sim$500 Myr up to~$\sim$10
Gyr for conditions in typical star forming disc galaxies.
In the dwarf galaxy regime, the length scales are short, star formation is slow
\citep[e.g.,][]{Hudson2015}, and the ISM velocities are turbulent
\citep{Dutta2009, Stilp2013, Schleicher2016}.
With this combination, instantaneous mixing should be a good approximation,
though we are unaware of any studies which address this observationally.
As long as the approximation is valid, then there should exist an evolutionary
track in chemical space (e.g., the~\afe-\feh~plane) about which the intrinsic
scatter is negligible compared to the measurement uncertainty.
This empirical test should be feasible on a galaxy-by-galaxy basis.
\par
With the goal of assessing the information content of one-zone GCE models
applied to dwarf galaxies, we emphasize that the accuracy of the methods we
outline in this paper are contingent on the validity of the instantaneous
mixing approximation.
This assumption reduces GCE to a system of coupled integro-differential
equations, which we solve using the publicly available~\textsc{Versatile
Integrator for Chemical Evolution} (\vice\footnote{
	\url{https://pypi.org/project/vice}
};~\citealp{Johnson2020}).
We provide an overview of the model framework below and refer to
\citet{Johnson2020} and the~\vice~science documentation\footnote{
	\url{https://vice-astro.readthedocs.io/en/latest/science_documentation}
} for further details.
\par
At a given moment in time, gas is added to the ISM via inflows and recycled
stellar envelopes and is removed from the ISM by star formation and outflows,
if present.
The sum of these terms gives rise to the following differential equation
describing the evolution of the gas supply:
\begin{equation}
\dot{M}_\text{g} = \dot{M}_\text{in} - \dot{M}_\star - \dot{M}_\text{out}
+ \dot{M}_\text{r},
\label{dga:eq:mdotgas}
\end{equation}
where~$\dot{M}_\text{in}$ is the infall rate,~$\dot{M}_\star$ is the SFR,
$\dot{M}_\text{out}$ is the outflow rate, and~$\dot{M}_\text{r}$ describes
the return of stellar envelopes from previous generations of stars.
\par
\vice~implements the same characterization of outflows as the~\textsc{FlexCE}
\citep{Andrews2017} and~\textsc{OMEGA}~\citep{Cote2017} chemical evolution
codes in which a ``mass-loading factor''~$\eta$ describes a linear relationship
between the outflow rate itself and the SFR:
\begin{equation}
\eta \equiv \frac{\dot{M}_\text{out}}{\dot{M}_\star}.
\label{dga:eq:massloading}
\end{equation}
This parametrization is appropriate for models in which massive stars are the
dominant source of energy for outflow-driving winds.
Empirically, the strength of outflows (i.e., the value of~$\eta$) is strongly
degenerate with the absolute scale of nucleosynthetic yields.
We discuss this further below and quantify the strength of the degeneracy in
more detail in Appendix~\ref{dga:sec:degeneracy}.
\par
The SFR and the mass of the ISM are related by the SFE timescale~$\tau_\star$,
defined as the ratio of the two:
\begin{equation}
\tau_\star \equiv \frac{M_\text{g}}{\dot{M}_\star}.
\label{dga:eq:taustar}
\end{equation}
The inverse~$\tau_\star^{-1}$ is the SFE itself, quantifying the
\textit{fractional} rate at which some ISM fluid element is forming stars.
Some authors refer to~$\tau_\star$ as the ``depletion time''
\citep[e.g.,][]{Tacconi2018} because it describes the e-folding decay timescale
of the ISM mass due to star formation if no additional gas is added.
Our nomenclature follows~\citet{Weinberg2017b}, who demonstrate that depletion
times in GCE models can shorten significantly in the presence of outflows.
\par
The recycling rate~$\dot{M}_\text{r}$ is a complicated function which depends
on the stellar initial mass function~\citep[IMF; e.g.,][]{Salpeter1955,
Miller1979, Kroupa2001, Chabrier2003}, the initial-final remnant mass relation
\citep[e.g.,][]{Kalirai2008}, and the mass-lifetime relation\footnote{
	We assume a~\citet{Kroupa2001} IMF and the~\citet{Larson1974} mass-lifetime
	relation throughout this paper.
	These choices do not significantly impact our conclusions as~$\eta$
	and~$\tau_\star$ play a much more significant role in establish the
	evolutionary histories of our GCE models.
	Our fitting method is nonetheless easily extensible to models which relax
	these assumptions.
}~\citep*[e.g.,][]{Larson1974, Maeder1989, Hurley2000}, all of which must then
be convolved with the SFH.
However, the detailed rate of return of stellar envelopes has only a
second-order effect on the gas-phase evolutionary track in the~\afe-\feh~plane.
The first-order details are instead determined by the SFE timescale~$\tau_\star$
and the mass-loading factor~$\eta$ (see discussion in~\citealt{Weinberg2017b}).
In the absence of sudden events such as a burst of star formation, the detailed
form of the SFH actually has minimal impact of the shape of the model track
\citep{Weinberg2017b, Johnson2020}.
That information is instead encoded in the stellar MDFs (i.e., the density of
stars along the track).
\par
In the present paper, we focus on the enrichment of the so-called ``alpha''
(e.g., O, Ne, Mg) and ``iron-peak'' elements (e.g., Cr, Fe, Ni, Zn), with the
distribution of stars in the~\afe-\feh~plane being our primary observational
diagnostic to distinguish between GCE models.
Massive stars and their core collapse SNe (CCSNe) are the dominant enrichment
source of alpha elements in the universe, while iron-peak elements are produced
in significant amounts by both massive stars and SNe Ia~\citep[e.g.,][]{Johnson2019}.
In detail, some alpha and iron-peak elements also have contributions from slow
neutron capture nucleosynthesis, an enrichment pathway responsible for much of
the abundances of yet heavier nuclei (specifically Sr and up).
Because the neutron capture yields of alpha and iron-peak elements are
small compared to their SN yields, we do not discuss this process further.
Our fitting method is nonetheless easily extensible to GCE models which do,
provided that the data contain such measurements.
\par
Due to the steep nature of the stellar mass-lifetime relation
\citep[e.g.,][]{Larson1974, Maeder1989, Hurley2000}, massive stars, their winds,
and their SNe enrich the ISM on~$\sim$few Myr timescales.
As long as these lifetimes is shorter than the relevant timescales for a
galaxy's evolution and the present-day stellar mass is sufficiently high such
that stochastic sampling of the IMF does not significantly impact the yields,
then it is adequate to approximate this nucleosynthetic material as some
population-averaged yield ejected instantaneously following a single stellar
population's formation.
This implies a linear relationship between the CCSN enrichment rate and the
SFR:
\begin{equation}
\dot{M}_\text{x}^\text{CC} = y_\text{x}^\text{CC} \dot{M}_\star,
\end{equation}
where~$y_\text{x}^\text{CC}$ is the IMF-averaged fractional net yield from
massive stars of some element x.
That is, for a fiducial value of~$y_\text{x}^\text{CC} = 0.01$, 100~\msun~of
star formation would produce 1~\msun~of~\textit{newly produced} element x (the
return of previously produced metals is implemented as a separate term
in~\vice; see~\citealt{Johnson2020} or the~\vice~science documentation for
details).
\par
Unlike CCSNe, SNe Ia occur on a significantly extended DTD.
The details of the DTD are a topic of active inquiry~\citep[e.g.,][]{Greggio2005,
Strolger2020, Freundlich2021}, and at least a portion of the uncertainty can be
traced to uncertainties in both galactic and cosmic SFHs.
Comparisons of the cosmic SFH~\citep[e.g.,][]{Hopkins2006, Madau2014, Davies2016,
Madau2017, Driver2018} with volumetric SN Ia rates as a function of redshift
indicate that the cosmic DTD is broadly consistent with a uniform~$\tau^{-1}$
power-law (\citealp{Maoz2012a};~\citealp*{Maoz2012b};~\citealp{Graur2013,
Graur2014}).
Following~\citet{Weinberg2017b}, we take a~$\tau^{-1.1}$ power-law DTD with a
minimum delay time of~$t_\text{D} = 150$ Myr, though in principle this
delay can be as short as~$t_\text{D} \approx 40$ Myr due to the lifetimes
of the most massive white dwarf progenitors.
For any selected DTD~$R_\text{Ia}(\tau)$, the SN Ia enrichment rate can be
expressed as an integral over the SFH weighted by the DTD:
\begin{equation}
\dot{M}_\text{x}^\text{Ia} = y_\text{x}^\text{Ia}\ddfrac{
	\int_0^{T - t_\text{D}} \dot{M}_\star(t) R_\text{Ia}(T - t) dt
}{
	\int_0^\infty R_\text{Ia}(t) dt
}.
\end{equation}
In general, the mass of some element x in the ISM is also affected by outflows,
recycling and star formation.
The total enrichment rate can be computed by simply adding up all of the source
terms and subtracting the sink terms:
\begin{equation}
\dot{M}_\text{x} = \dot{M}_\text{x}^\text{CC} + \dot{M}_\text{x}^\text{Ia} -
Z_\text{x}\dot{M}_\star - Z_\text{x}\dot{M}_\text{out} + \dot{M}_{\text{x,r}},
\label{dga:eq:enrichment}
\end{equation}
where~$Z_x = M_\text{x} / M_\text{ISM}$ is the abundance by mass of the nuclear
species x in the ISM.
This equation as written assumes that the outflowing material is of the same
composition as the ISM, but in principle, the various nuclear species of
interest may be some factor above or below the ISM abundance.
In the present paper we assume all accreting material to be zero metallicity
gas; when this assumption is relaxed, an additional term
$Z_\text{x,in}\dot{M}_\text{in}$ appears in this equation.
\par
As mentioned above, the strength of outflows is degenerate with the absolute
scale of nucleosynthetic yields.
This ``yield-outflow degeneracy'' is remarkably strong, and it arises because
yields and outflows are the dominant source and sink terms in equation
\refp{dga:eq:enrichment} above.
As a consequence, high-yield and high-outflow models generally have a
low-yield and low-outflow counterpart that predicts a similar enrichment
history.
In order to break this degeneracy, only a single parameter setting the absolute
scale is required.
To this end, we set the alpha element yield from massive stars to be exactly
$\yacc = 0.01$ and let our Fe yields be free parameters.
Appropriate for O, this value is loosely motivated by nucleosynthesis theory in
that massive star evolutionary models (e.g.,~\citealp*{Nomoto2013};
\citealp{Sukhbold2016, Limongi2018}) typically predict
$y_\text{O}^\text{CC} = 0.005 - 0.015$ (see discussion in, e.g.,
\citealp{Weinberg2017b} and~\citealp{Johnson2020}).
This value is~$\sim$1.75 times the solar O abundance of~$\sim$0.57\%
\citep{Asplund2009}, and if we had chosen a different alpha element (e.g., Mg),
then we would need to adjust accordingly to account for the intrinsically lower
abundance (e.g., $\yacc = 1.75 Z_{\text{Mg},\odot} \approx
\scinote{1.2}{-4}$).\footnote{
	The lighter alpha elements like O and Mg evolve similarly in GCE models due
	to metallicity-independent yields dominated by massive stars, so it is
	mathematically convenient to treat them as a single nuclear species
	under the assertion that [O/Mg]~$\approx 0$ (this assumption is indeed
	supported by empirical measurements in APOGEE; see, e.g., Fig. 8 of
	\citealt{Weinberg2019}).
	In practice, however, we use the~$\yacc = 0.01$ value for O and a solar
	abundance of~$Z_{\text{O},\odot} = 0.00572$~\citep{Asplund2009}.
}
The primary motivation behind this choice is to select a round number that
allows our best-fit values affected by this degeneracy to be scaled up or down
under different assumptions regarding the scale of effective yields.
We reserve further discussion of this topic for Appendix~\ref{dga:sec:degeneracy}
where we also quantify the considerably strength of the yield-outflow
degeneracy in more detail.

