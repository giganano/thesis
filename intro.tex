
\documentclass[main.tex]{subfiles}
\begin{document}
\begin{doublespace}

\chapter{Introduction}
\label{main:sec:intro}

\textit{``We are made of star stuff.''}
Carl Sagan meant it literally in this famous quote.
When the universe was only a few minutes old, nuclear fusion in the hot plasma
left behind only hydrogen, helium, and trace amounts of lithium.
Originally proposed by the landmark work of~\citet{Alpher1948}, this process
of~\textit{Big Bang Nucleosynthesis} produced only the lightest elements on the
periodic table.
Everything else, from the carbon in our bodies and the oxygen we breathe to the
silicon in our computers and the iron in our infrastructure, was synthesized in
stars and supernovae.
\par
Supported against their own gravity by nuclear fusion in their hot, dense cores,
stars change their own chemical compositions over time.
As stars approach the ends of their lives, they fuse progressively heavier
nuclei before either ejecting their outer envelopes or exploding as supernovae.
Early studies of the origin of the elements applied the principles of nuclear
physics to the astrophysical conditions in which fusion reactions arise, a
review of which can be found in the seminal work of~\citet*{Burbidge1957}.
This connection became the foundation of stellar nucleosynthesis, which
describes the origin of the remaining stable elements on the periodic table.
\par
Being the sites of star formation in the universe, galaxies facilitate this
whole process.
When a star dies and disperses its envelope back to the interstellar medium
(ISM), the material is composed of heavier nuclei than when the star formed.
This material then mixes with the ISM gas, and once cooled, it can be
incorporated into new stars, which then inherit the chemical composition of
their natal gas clouds.
Through this so-called~\textit{baryon cycle}, galaxies are like petri dishes of
their own nuclear reactions.
In the same way that various microbes produce different outcomes when grown in
a petri dish, the various evolutionary pathways traced by galaxies produce
different chemical compositions.
\par
This connection between galaxies, stars, and nuclear reactions is the basis of
galactic chemical evolution (GCE).
The GCE ansatz combines stellar birth rates with the initial mass distribution
of stars~\citep[e.g.,][]{Salpeter1955, Kroupa2001, Chabrier2003}, their
lifetimes~\citep[e.g.,][]{Larson1974, Maeder1989, Hurley2000}, and elemental
yields due to nuclear fusion~\citep[e.g.,][]{Woosley1995, Iwamoto1999,
Karakas2016} to estimate production rates of various nuclear species over a
galaxy's lifetime.
The simplest GCE models are commonly referred to as~\textit{one-zone} models,
which assume that elemental yields from stars and supernovae are instantaneously
mixed through the ISM and are immediately available for star formation.
This approximation reduces GCE to a system of coupled integro-differential
equations.
\par
Early investigations with one-zone models described the Milky Way as a
``closed-box,'' in which there was no ongoing accretion or ejection of material
and all of its mass was present and available for star formation at the time of
its initial collapse.
These models predicted what would come to be known as the ``G-dwarf problem,''
in which the number of stars with metal abundances greater than that of the Sun
was drastically overpredicted.
The solution arose by relaxing the closed-box assumption, providing an early
indication that ongoing accretion and ejection play significant roles in
shaping a galaxy's chemical enrichment history~\citep[e.g.,][]{Larson1972,
Chiosi1980}.

\section{Spectroscopic Surveys}
\label{main:sec:intro:surveys}

The detailed chemical compositions of stars can be measured with
\textit{spectroscopy}, which breaks up light into its component wavelengths.
Due to the quantized nature of electron energy levels, atoms emit and absorb
light at specific wavelengths that can be measured in a laboratory.
Measurements of a star's chemical composition are then feasible by comparing
these absorption and emission lines in a star's spectrum with radiative
transfer models of its photosphere.
Statistical samples of stars, their abundances, and their ages then provide
empirical tests with which to distinguish between GCE models and shed light on
a galaxy's evolutionary history.
\par
Modern spectroscopic surveys, such as LAMOST\footnote{
	Large sky Area Multi-Object fibre Spectroscopic Telescope.
}\space\citep{Luo2015}, GALAH\footnote{
	GALactic Archaeology with Hermes.
}\space\citep{DeSilva2015, Martell2017},~\gaia-ESO\footnote{
	European Southern Observatory.
}\space\citep{Gilmore2012}, and
APOGEE\footnote{
	Apache Point Observatory Galaxy Evolution Experiment, conducted as part of
	the third~\citep{Eisenstein2011} and fourth~\citep{Blanton2017} iteration
	of the Sloan Digital Sky Survey.
}\space\citep{Majewski2017}, have mapped the chemical abundance structure of
the Milky Way in unprecedented detail.
APOGEE in particular has achieved excellent coverage, with measurements of 15
or more elemental abundances in over 650,000 stars in the Galactic disk, bulge,
and halo in its final data release~\citep{Abdurrouf2022}.
This depth is achieved primarily by targeting luminous red giant stars
accessible at large distances~\citep{Zasowski2013, Zasowski2017, Beaton2021,
Santana2021} and measuring their spectra at near-infrared wavelengths, which
are less susceptible to dust obscuration
($1.51 - 1.70~\mu$m;~\citealp{Wilson2019}).
\par
APOGEE has also measured spectra of resolved stars in the Milky Way's satellite
dwarf galaxies, such as the Large and Small Magellanic Clouds, Fornax, and the
Sagittarius Dwarf Galaxy~\citep[see, e.g.,][]{Hasselquist2021}.
While some of these systems have remained mostly intact, others have been
disrupted and now make up the Milky Way's~\textit{stellar halo}.
This diffuse, kinematically incoherent population is comprised in small part by
stars that formed when the Milky Way disk initially collapsed gravitationally
(i.e., the~\textit{in-situ} halo), but mostly by these accreted populations
from dwarf galaxies now tidally disrupted~\citep[e.g.,][]{Naidu2020}.
In addition to APOGEE, the H3 survey\footnote{
	Hectochelle in the Halo at High resolution.
}\space\citep{Conroy2019}, whose science goals focus on the stellar halo
specifically, has played a key role in shaping the community's understanding of
this recent hierarchical assembly.

\section{Stellar Yields}
\label{main:sec:intro:yields}

Since they quantify the amount of any given element produced by a generation of
stars, yields are arguably the most important parameters in GCE models.
Despite their central role, these quantities are poorly understood.
Much of the uncertainty arises because nucleosynthesis is highly sensitive to
stellar evolution processes that are also poorly understood, such as
rotationally induced mixing~\citep[e.g.,][]{Frischknecht2016}, convection
\citep[e.g.,][]{Chieffi2001}, and the supernova explosion mechanisms
themselves~\citep[e.g.,][]{Griffith2021b}.
\par
GCE models conventionally adopt a set of stellar yields and hold them fixed,
which introduces a source of systematic uncertainty.
To date, few studies have systematically investigated the impact of yield
assumptions in GCE models to calibrate parameterizations against abundance
trends and assess potential implications for stellar evolution
\citep[e.g.,][]{delosReyes2022, Womack2023}.
{\color{red} Needs a conclusion of some sort.}


\section{Dwarf Galaxies and the Stellar Halo}
\label{main:sec:intro:dwarfs}

Dwarf galaxies are much more numerous than massive spirals like the Milky Way
and Andromeda~\citep[e.g.,][]{Bell2003, Baldry2012}.
In the local universe, their prevalence allows them to be studied in
statistical samples with resolved stars.
GCE models of these systems are straightforward as classical one-zone models
are able to explain the data reasonably well~\citep[see, e.g.,][]{Kirby2011,
delosReyes2022}.
These systems tend to have bursty, episodic star formation histories
\citep[e.g.,][]{Weisz2014a} that are truncated, or ``quenched'' due to cosmic
background radiation or an interaction with a nearby, more massive galaxy like
the Milky Way~\citep[e.g.,][]{Weisz2014b, Weisz2015, Naidu2022}.
\par
{\color{red}
Something about how bursts in star formation should impact their chemical
abundance structure.
Then something about how the cessation in star formation should leave an
imprint in the abundances of the stars -- where the metallicity distributions
cut off.
}

\section{The Milky Way Disk}
\label{main:sec:intro:mw}

Our location within the Milky Way is both a blessing and a curse.
On the one hand, the Galaxy can be studied with many more resolved stars than
can be measured in external galaxies.
On the other hand, dust obscuration makes it difficult to access far away stars
in the plane of the Galaxy.
Empirically determining the Milky Way's detailed morphology remains a topic of
active inquiry in the literature~\citep[e.g.,][]{Nataf2013, Bovy2019,
Shen2020}, a problem which would be solved quickly if we could simply observed
the Galaxy from the outside.
\par
GCE models of the Milky Way are inherently more complicated than in dwarf
galaxies.
While early investigations used one-zone models to describe the Galaxy,
dynamical interactions with the bar and spiral arms can change the orbital
radii of stars and carry them to a region of the Galaxy far from where they
formed, a process known as~\textit{stellar migration}
\citep[e.g.,][]{Sellwood2002, Schoenrich2009a}.
This effect prompted the development of so-called~\textit{multi-zone} models,
which add spatial information back into the instantaneous mixing ansatz by
coupling a series of one-zone models through the exchange of stars
\citep[e.g.,][]{Matteucci1989, Wyse1989, Prantzos1995}.
\par
Results within this paradigm have provided conflicting results.
Models have been able to reproduce various characteristics of the disk abundance
structure with both continuous~\citep[e.g.,][]{Chen2023} and episodic star
formation histories~\citep[e.g.,][]{Spitoni2021}.
Distinguishing between these models is challenging due to their dependence on
many uncertain processes (e.g., nucleosynthetic yields, stellar migration, the
star formation history of the Galaxy).
Measurement uncertainty exacerbates these difficulties; stellar ages are
particularly uncertain~\citep[e.g.,][]{Soderblom2010, Chaplin2013}, and
spectral synthesis is often a source of considerable systematic error
\citep[e.g.,][]{Joensson2018, Eilers2022}.
Suffice it to say that the evolutionary history of our own Galaxy remains an
enigma.

\section{Bridging the Gap: Trends with Galaxy Mass}
\label{main:sec:inro:snrates}

The discussion in~\S~\ref{main:sec:intro:mw} and~\S~\ref{main:sec:intro:dwarfs}
above shed light on variations in chemical abundances between dwarf galaxies
and the Milky Way, indicative of correlations with galaxy mass.
In particular, the mass-metallicity relation, the mass-metallicity gradient
relation, the mass-metallicity-$\alpha$ relation.
SN surveys have shed light on the potential origin of some of these relations,
finding higher specific rates at low stellar masses.
An understanding of these rates should in principle have implications for
enrichment histories as a function of galaxy mass.

\section{The Scope of this Dissertation}
\label{main:sec:intro:scope}

The present era of large data sets provided by high-resolution spectroscopic
surveys and increasingly powerful computational tools enables us to reveal the
enrichment histories of the Milky Way and nearby dwarf galaxies in
unprecedented detail.
What are the evolutionary histories of galaxies, and how does this relate to
their observed and intrinsic properties?
What are the relative yields of different elements from stars and supernovae?
GCE models contain information on all of these processes.
\par
In this dissertation, I combine powerful theoretical and observational tools
to improve the understanding of galaxy evolution through abundance modeling.
In Chapter blah I do blah blah blah, and in the next chapter blah I do a
different blah blah blah.
Check it out!

\end{doublespace}
\end{document}

