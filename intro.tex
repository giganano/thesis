
\documentclass[main.tex]{subfiles}
\begin{document}
\begin{doublespace}

\chapter{Introduction}
\label{main:sec:intro}

\textit{``We are made of star stuff.''}
Carl Sagan meant it literally in this famous quote.
When the universe was only a few minutes old, nuclear fusion in the hot plasma
left behind only hydrogen, helium, and trace amounts of lithium.
Originally proposed by the landmark work of~\citet{Alpher1948}, this process
of~\textit{Big Bang Nucleosynthesis} produced only the lightest elements on the
periodic table.
Everything else, from the carbon in our bodies and the oxygen we breathe to the
silicon in our computers and the iron in our infrastructure, was synthesized in
stars and supernovae.
\par
Supported against their own gravity by nuclear fusion in their hot, dense cores,
stars change their own chemical compositions over time.
As stars approach the ends of their lives, they fuse progressively heavier
nuclei before either ejecting their outer envelopes or exploding as supernovae.
Early studies of the origin of the elements applied the principles of nuclear
physics to the astrophysical conditions in which fusion reactions arise, a
review of which can be found in the seminal work of~\citet*{Burbidge1957}.
This connection became the foundation of stellar nucleosynthesis, which
describes the origin of the remaining stable elements on the periodic table.
\par
Being the sites of star formation in the universe, galaxies facilitate this
whole process.
When a star dies and disperses its envelope back to the interstellar medium
(ISM), the material is composed of heavier nuclei than when the star formed.
This material then mixes with the ISM gas, and once cooled, it can be
incorporated into new stars, which then inherit the chemical composition of
their natal gas clouds.
Through this so-called~\textit{baryon cycle}, galaxies are like petri dishes of
their own nuclear reactions.
In the same way that various microbes produce different outcomes when grown in
a petri dish, the various evolutionary pathways traced by galaxies produce
different chemical compositions.
\par
This connection between galaxies, stars, and nuclear reactions is the basis of
galactic chemical evolution (GCE).
The GCE ansatz combines stellar birth rates with the initial mass distribution
of stars~\citep[e.g.,][]{Salpeter1955, Kroupa2001, Chabrier2003}, their
lifetimes~\citep[e.g.,][]{Larson1974, Maeder1989, Hurley2000}, and elemental
yields due to nuclear fusion~\citep[e.g.,][]{Woosley1995, Iwamoto1999,
Karakas2016} to estimate production rates of various nuclear species over a
galaxy's lifetime.
The simplest GCE models are commonly referred to as~\textit{one-zone} models,
which assume that elemental yields from stars and supernovae are instantaneously
mixed through the ISM and are immediately available for star formation.
This approximation reduces GCE to a system of coupled integro-differential
equations.
\par
Early investigations with one-zone models described the Milky Way as a
``closed-box,'' in which there was no ongoing accretion or ejection of material
and all of its mass was present and available for star formation at the time of
its initial collapse.
These models predicted what would come to be known as the ``G-dwarf problem,''
in which the number of stars with metal abundances greater than that of the Sun
was drastically overpredicted.
The solution arose by relaxing the closed-box assumption, providing an early
indication that ongoing accretion and ejection play significant roles in
shaping a galaxy's chemical enrichment history~\citep[e.g.,][]{Larson1972,
Chiosi1980}.

\section{Spectroscopic Surveys}
\label{main:sec:intro:surveys}

The detailed chemical compositions of stars can be measured with
\textit{spectroscopy}, which breaks up light into its component wavelengths.
Due to the quantized nature of electron energy levels, atoms emit and absorb
light at specific wavelengths that can be measured in a laboratory.
Measurements of a star's chemical composition are then feasible by comparing
these absorption and emission lines in a star's spectrum with radiative
transfer models of its photosphere.
Statistical samples of stars, their abundances, and their ages then provide
empirical tests with which to distinguish between GCE models and shed light on
a galaxy's evolutionary history.
\par
Modern spectroscopic surveys, such as LAMOST\footnote{
	Large sky Area Multi-Object fibre Spectroscopic Telescope.
}\space\citep{Luo2015}, GALAH\footnote{
	GALactic Archaeology with Hermes.
}\space\citep{DeSilva2015, Martell2017},~\gaia-ESO\footnote{
	European Southern Observatory.
}\space\citep{Gilmore2012}, and
APOGEE\footnote{
	Apache Point Observatory Galaxy Evolution Experiment, conducted as part of
	the third~\citep{Eisenstein2011} and fourth~\citep{Blanton2017} iteration
	of the Sloan Digital Sky Survey.
}\space\citep{Majewski2017}, have mapped the chemical abundance structure of
the Milky Way in unprecedented detail.
APOGEE in particular has achieved excellent coverage, with measurements of 15
or more elemental abundances in over 650,000 stars in the Galactic disk, bulge,
and halo in its final data release~\citep{Abdurrouf2022}.
This depth is achieved primarily by targeting luminous red giant stars
accessible at large distances~\citep{Zasowski2013, Zasowski2017, Beaton2021,
Santana2021} and measuring their spectra at near-infrared wavelengths, which
are less susceptible to dust obscuration
($1.51 - 1.70~\mu$m;~\citealp{Wilson2019}).
\par
APOGEE has also measured spectra of resolved stars in the Milky Way's satellite
dwarf galaxies, such as the Large and Small Magellanic Clouds, Fornax, and the
Sagittarius Dwarf Galaxy~\citep[see, e.g.,][]{Hasselquist2021}.
While some of these systems have remained mostly intact, others have been
disrupted and now make up the Milky Way's~\textit{stellar halo}.
This diffuse, kinematically incoherent population is comprised in small part by
stars that formed when the Milky Way disk initially collapsed gravitationally
(i.e., the~\textit{in-situ} halo), but mostly by these accreted populations
from dwarf galaxies now tidally disrupted~\citep[e.g.,][]{Naidu2020}.
In addition to APOGEE, the H3 survey\footnote{
	Hectochelle in the Halo at High resolution.
}\space\citep{Conroy2019}, whose science goals focus on the stellar halo
specifically, has played a key role in shaping the community's understanding of
this recent hierarchical assembly.

\section{Stellar Yields}
\label{main:sec:intro:yields}

Yields quantify the amount of a given element produced by stars, making them
arguably the most fundamental parameter in GCE models.
Despite their central role, these quantities are poorly understood.
Much of the uncertainty arises as a consequence of uncertainties in stellar
evolution.
Nuclear reactions and their rates are highly sensitive to mixing processes
induced by stellar rotation~\citep[e.g.,][]{Frischknecht2016}, convection
\citep[e.g.,][]{Chieffi2001}, and the supernova explosion mechanisms
themselves~\citep[e.g.,][]{Griffith2021b}.
\par
Conventionally, GCE models adopt a set of stellar yields and hold them fixed.
A source of systematic uncertainty is introduced when additional assumptions
regarding stellar yields are not considered.
To date, few studies have systematically investigated the impact of yield
assumptions in GCE models to calibrate parameterizations against abundance
trends and assess potential implications for stellar evolution
\citep[e.g.,][]{delosReyes2022, Womack2023}.
The trends of metal abundances with, e.g., other metals, stellar age, and
galaxy mass contain a wealth of information on how and where various nuclear
species are produced, which can be illuminated with GCE models.
These empirical calibrations of elemental yields from stars could have
powerful implications for stellar evolution.

\section{Dwarf Galaxies and the Stellar Halo}
\label{main:sec:intro:dwarfs}

Dwarf galaxies are much more numerous than massive spirals like the Milky Way
and Andromeda~\citep[e.g.,][]{Bell2003, Baldry2012}.
In the local universe, their prevalence and proximity allow them to be studied
in statistical samples with resolved stars.
GCE models of these systems are straightforward as classical one-zone models
are able to explain the data reasonably well~\citep[see, e.g.,][]{Kirby2011,
delosReyes2022}.
These systems tend to have bursty, episodic star formation histories
\citep[e.g.,][]{Weisz2014a} that are truncated, or ``quenched'' either due to
cosmic background radiation or interactions with a nearby, more massive galaxy
like the Milky Way~\citep[e.g.,][]{Weisz2014b, Weisz2015, Naidu2022}.
\par
As different evolutionary pathways imprint differently on the chemical
compositions of galaxies, these mechanisms should in principle imprint on the
chemical abundances of stars.
Sudden events, such as bursts in star formation, are indeed predicted to leave
distinct signatures in the elemental abundances of stellar populations by
perturbing the number of stars producing various elements~\citep{Weinberg2017b}.
Nonetheless, quantitative questions remain.
What potential observational diagnostics of these events might be feasible
given a sample of stars with elemental abundance measurements?
Would such a sample contain information on the duration of star formation in
these galaxies?

\section{The Milky Way Disk}
\label{main:sec:intro:mw}

Our location within the Milky Way is both a blessing and a curse.
On the one hand, the Galaxy can be studied with many more resolved stars than
can be measured in external galaxies.
On the other hand, dust obscuration makes it difficult to access far away stars
in the plane of the Galaxy.
Empirically determining the Milky Way's detailed morphology remains a topic of
active inquiry in the literature~\citep[e.g.,][]{Nataf2013, Bovy2019,
Shen2020}, a problem which would be solved quickly if we could simply observed
the Galaxy from the outside.
\par
GCE models of the Milky Way are inherently more complicated than in dwarf
galaxies.
While early investigations used one-zone models to describe the Galaxy,
dynamical interactions with the bar and spiral arms can change the orbital
radii of stars and carry them to a region of the Galaxy far from where they
formed, a process known as~\textit{stellar migration}
\citep[e.g.,][]{Sellwood2002, Schoenrich2009a}.
This effect prompted the development of so-called~\textit{multi-zone} models,
which add spatial information back into the instantaneous mixing ansatz by
coupling a series of one-zone models through the exchange of stars
\citep[e.g.,][]{Matteucci1989, Wyse1989, Prantzos1995}.
\par
Results within this paradigm have provided conflicting results.
Models have been able to reproduce various characteristics of the disk abundance
structure with both continuous~\citep[e.g.,][]{Chen2023} and episodic star
formation histories~\citep[e.g.,][]{Spitoni2021}.
Distinguishing between these models is challenging due to their dependence on
many uncertain processes (e.g., nucleosynthetic yields, stellar migration, the
star formation history of the Galaxy).
Measurement uncertainty exacerbates these difficulties; stellar ages are
particularly uncertain~\citep[e.g.,][]{Soderblom2010, Chaplin2013}, and
spectral synthesis is often a source of considerable systematic error
\citep[e.g.,][]{Joensson2018, Eilers2022}.
Suffice it to say that the evolutionary history of our own Galaxy remains an
enigma.

\section{Bridging the Gap: Trends with Galaxy Mass}
\label{main:sec:inro:snrates}

As one might suspect given the obvious differences between galaxies like the
Milky Way and nearby dwarf galaxies (e.g., mass, size), the two classes of
systems assemble their mass qualitatively differently.
Pursuing the petri dish analogy once more, the differences in evolutionary
pathways should leave an imprint in the chemical compositions of vastly
different galaxies.
Such a suspicion would indeed be confirmed by empirical investigations.
\par
First and foremost, more massive galaxies tend to host both gas reservoirs
\citep{Tremonti2004, Zahid2011, Zahid2012, Andrews2013} and stellar populations
\citep{Gallazzi2005, Kirby2013} with a higher metal content than their low-mass
counterparts.
Previous work has argued that metal-enriched outflows drive the slope at the
low-mass end~\citep{Finlator2008, Peeples2011, Chisholm2018}.
The relationship flattens off for high-mass galaxies, which~\citet{Zahid2014}
link to the stellar-to-gas mass ratios in galaxies.
Weaker correlations with stellar mass have recently been uncovered thanks to
modern surveys like~\gaia~\citep{GaiaCollaboration2016} and H3
\citep{Conroy2019}.
For example, sufficiently large galaxies have metallicity gradients, whereby
the inner regions are higher metallicity on average than the outer regions,
and the slope of this radial trend seems to steepen with galaxy mass
\citep[e.g.,][]{Goddard2017}.
\par
Of particular interest to GCE is the relationship between galaxy mass and
supernova rates.
Observationally, dwarf galaxies are observed to have considerably higher rates
of Type Ia supernovae per unit mass than their larger counterparts
\citep[e.g.,][]{Leaman2011, Li2011}.
The first investigations into the origin of this result indicated that the
variations in star formation histories with galaxy mass can explain the
results~\citep[e.g.,][]{Graur2013, Graur2015}.
However, a larger sample allowed~\citet{Brown2019} to push these measurements
to lower galaxy masses than previous work ($\sim$$10^7$~\msun), revealing that
the specific rates continued to increase over an entire order of magnitude.
Though the exact slope of the empirically derived trend with mass is subject to
uncertainties surrounding the stellar mass distribution of galaies (see
discussion in, e.g.,~\citealt{Gandhi2022}), an additional effect seems to be
required regardless of its exact strength.
\par
It is imperative that the ingredients and predictions of GCE models be
consistent with the results of supernova surveys.
As the source of a considerable fraction of the metals in the universe, these
data constitute an additional empirical testing ground for enrichment rates
aside from abundances themselves.
What additional processes could impact supernova rates in dwarf galaxies?
How strong must these effects be to explain the empirical trends?
Additional investigations are required to make sense of
\citet{Brown2019}'s~\citeyearpar{Brown2019} realization.

\section{The Scope of this Dissertation}
\label{main:sec:intro:scope}

The present era of large data sets provided by spectroscopic surveys and
increasingly powerful computational tools enables us to reveal the enrichment
histories of the Milky Way and nearby dwarf galaxies in unprecedented detail.
These measurements contain not only information on the assembly and evolutionary
histories of galaxies, but the stellar evolution processes which formed these
nuclei as well.
% These processes, which span many orders of magnitude in physical scale, can be
% understood through GCE models.
\par
In this dissertation, I take a journey across the mass range of galaxies with
chemical evolution models.
I start with dwarf galaxies, which have been described sufficiently accurately
with classic, one-zone GCE models.
In Chapter~\ref{bursts}, I consider the bursty star formation histories
of these low-mass systems~\citep[e.g.,][]{Weisz2014a}, assessing the
observational clues that should be left behind by these evolutionary chennels.
As different elements trace different nucleosynthetic pathways within stars
\citep[e.g.,][]{Johnson2019}, I consider a range of elements to shed light
on as many empirical clues as possible.
\par
To deduce the details of galaxy evolution at the low-mass end quantitatively, I
then develop statistical tools with which to determine best-fit
parameters of one-zone models in Chapter~\ref{dga}.
I validate this method with mock data samples to ensure that the methods
employed do not introduce systematic errors in the inferred evolutionary
parameters.
I introduce stellar yields as free parameters in this framework, both to
demonstrate the ability of GCE models to deduce nucleosynthetic pathways and
to minimize systematic uncertainties.
I then apply this method to two disrupted dwarf galaxies in the stellar halo
of the Milky Way:~\gaia-Sausage Enceladus~\citep{Helmi2018,
Belokurov2018} and Wukong/LMS-1~\citep{Naidu2020, Naidu2022, Yuan2020}.
\par
Having already zoomed out to consider a range of stellar masses in
Chapter~\ref{dga}, I then investigate Type Ia supernova rates across
galaxy mass in Chapter~\ref{iarates}.
I combine a state-of-the-art theoretical model describing the star
formation histories of galaxies in the cosmological context
\citep{Behroozi2019} with standard parameterizations of the delay-time
distributions of these events~\citep[e.g.,][]{Maoz2012a, Maoz2012b}.
As Type Ia supernovae arise from binary systems~\citep{Whelan1973, Iben1984,
Webbink1984}, I take special consideration of the metallicity dependence
of the binary fraction~\citep{Badenes2018, Moe2019}.
\par
I then turn my sights toward home and investigate the enrichment history
of our own Milky Way in Chapter~\ref{migration}.
Due to the size of the Galaxy, the models need updated to account for the
effects of evolving stellar orbits~\citep[e.g.,][]{Sellwood2002,
Schoenrich2009a}.
I focus on assessing which empirical results can be explained by standard
expectations of the Galaxy's evolutionary history and simple variations thereof
combined with this process of radial migration.
\par
To further demonstrate the utility of galaxies as laboratories of stellar
evolution, in Chapter~\ref{ohno} I use these models of the Milky Way
to deduce the origins of nitrogen nucleosynthesis, a particular challenging
element to understand.
Nitrogen production is most efficient in asymptotic giant branch undergoing
both third dredge-up and hot bottom burning, which are poorly understood
processes (see discussion in, e.g.,~\S~5 of~\citealt{Karakas2016}).
Accurate and reliable empirical calbirations of nitrogen yields could therefore
have important implications for stellar evolution, particularly for
rotation-induced mixing processes~\citep[e.g.,][]{Heger2010, Frischknecht2016}.
\par
In Chapter~\ref{outflows}, I use a large sample of APOGEE stars to measure the
evolution of the Galactic radial metallicity gradient by conditioning stars on
their ages as inferred from a Bayesian convolutional neural network.
I use the GCE models from previous chapter to qualitatively interpret the
origin of this ubiquitous feature of spiral galaxies.
I discuss the implications of the interpretation that the gradient arose out of
differences in chemical equilibrium abundances at fixed stellar age.
\par
Lastly, I summarize my conclusions from this dissertaion and discuss my
future plans to deepen the community's understanding of galaxy evolution and
stellar nucleosynthesis in Chapter~\ref{conclusions}.

\end{doublespace}
\end{document}

