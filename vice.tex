
\documentclass[main.tex]{subfiles}
\begin{document}

\chapter{\vice}
\label{vice}

The~\textsc{Versatile Integrator for Chemical Evolution}
(\textsc{VICE})\footnote{
	Install (PyPI):~\url{https://pypi.org/project/vice}
	\\
	Documentation:~\url{https://vice-astro.readthedocs.io}
	\\
	Source Code:~\url{https://github.com/giganano/VICE.git}
}~is an open-source~\python~package available for Linux and Mac OS X developed
as part of this dissertation.
The original version, developed for the purposes of the work in
Chapter~\ref{bursts}, included one-zone model integration features and the
routines for computing population-averaged yields from stellar populations.
\vice~has seen several updates over the course of this work, the largest of
which is the development of multi-zone model capabilities in
Chapter~\ref{migration}.
\par
The latest version (1.3.0) requires~\python~version 3.6.0 or higher and can be
installed from the Unix command line via~\texttt{python -m pip install vice}.
New users can launch a tutorial to familiarize themselves with~\vice~by then
running~\texttt{python -m vice -{}-tutorial}, and online documentation can be
launched with~\texttt{python -m vice -{}-docs}.
The~\python~code which integrates the models and produces the figures in
Chapters~\ref{bursts} and~\ref{migration} is available alongside~\vice's source
code in its GitHub repository.
\par
With an emphasis on versatility,~\vice~allows users to specify arbitrary
functions of time to describe various evolutionary parameters, including but
not limited to the star formation and infall histories, star formation
efficiency, mass-loading of outflows, the SN Ia delay-time distribution, and
element-by-element infall metallicities.
Yields from supernovae and asymptotic giant branch stars as well as the
stellar initial mass function can be speicifed as user-defined functions of
mass and metallicity.
While providing this level of flexibility in a~\python~package,~\vice~also
enjoys a backend implemented in C, providing it with the powerful computing
speeds of a compiled library.
\par
In Chapter~\ref{migration}, I extended the features implemented for
Chapters~\ref{bursts} and~\ref{dga} with an implementation of multi-zone
GCE models (see discussion in~\S~\ref{migration:sec:methods}).
Implemented with as purely generic of an algorithm as possible, these features
simply couple a series of one-zone models via the exchange of gas and stars.
For the sake of user-friendliness, it also includes an extension of these
capabilities specifically catered for the multi-ring class of models discussed
in Chapter~\ref{migration}.
\vice~requires~$\sim$2.5 CPU-hours per chemical element to integrate the
models in Chapter~\ref{migration}, in which it computes masses, abundances, and
initial and final radii for~$\sim$1.6 million stellar populations spanning a
little over 1,300 timesteps.
This estimate was made with a 3 GHz processor.
The outputs of these models require only~$\sim$235 MB of disc space each,
though when integrating they can occupy up to 3 GB of RAM due to the number of
stellar populations and timesteps used, both of which are adjustable
parameters.
\par
In Chapter~\ref{ohno}, we made a handful of updates.
They are as follows:
\begin{enumerate}

	\item While previously only a single power-law relationship between a
	star's mass and its lifetime were used, the options were expanded as this
	approximation underestimates the lifetimes of high-mass stars.
	We added:
	\begin{itemize}

		\item \citet{Larson1974} (\textbf{default})
		
		\item \citet{Maeder1989}

		\item \citet{Padovani1993}

		\item \citet{Kodama1997}

		\item \citet{Hurley2000}

		\item \citet{Vincenzo2016b}

	\end{itemize}
	Generally, chemical evolution models make similar predictions with any of
	these models as they are significantly different from one another.
	The default choice from~\citet{Larson1974} is a metallicity-independent
	parabola in~$\log \tau - \log m$ space for which we take the updated
	coefficients from~\citet{Kobayashi2004} and~\citet{David1990}.

	\item Two new sets of tables of yields from asymptotic giant branch stars
	were added.
	The~\karakas~and~\ventura~models were new to~\vice~(see discussion
	in~\S~\ref{ohno:sec:yields}).

	\item We added Type Ia supernova yields from sub-Chandrasekhar mass
	carbon-oxygen white dwarf models at various metallicities from
	\citet{Gronow2021a, Gronow2021b}.

\end{enumerate}

\end{document}
























