
\section{Discussion and Conclusions}
\label{iarates:sec:conclusions}

Building on LOSS~\citep{Li2011},~\citet{Brown2019} and~\citet{Wiseman2021}
found the SN Ia rates rise steeply toward low stellar mass.
The exact slope depends on the adopted SMF, with
$\dot{\text{N}} / \mstar \sim \mstar^{-0.3}$ for~\citet{Baldry2012} and
$\dot{\text{N}} / \mstar \sim \mstar^{-0.5}$ for~\citet{Bell2003}.
To explain this scaling with mass, we use the mean SFHs of galaxies predicted
by the~\um~\citep{Behroozi2019} semi-analytic model of galaxy formation, a
standard~$\tau^{-1}$ DTD~\citep[e.g.,][]{Maoz2012a}, and the empirical MZR as
parametrized by~\citet{Zahid2014} to relate stellar mass to metallicity and
build-in a~$Z^\gamma$ SN Ia rate dependence.
Our results depend only on the shape of the MZR and not its absolute
calibration.
While lower mass galaxies have younger stellar populations, we find that this
accounts for only a factor of~$\sim$2 increase in the specific rate between
$\mstar = 10^{7.2}$ and~$10^{10}~\msun$.
We can match the~$\mstar^{-0.3}$ increase if~$\gamma \approx -0.5$,
but~$\gamma \approx -1.5$ is required to explain the steeper
$\dot{\text{N}} / \mstar \sim \mstar^{-0.5}$ scaling.
\par
A scaling of~$\gamma = -0.5$ is in excellent agreement with the dependence of
the close binary fraction measured in APOGEE, which increases from~$\sim$10\%
at~$\sim$$3Z_\odot$ to~$\sim$40\% at~$\sim$$0.1Z_\odot$~\citep{Moe2019}.
This close match suggests that if a scaling of
$\dot{\text{N}}_\text{Ia} / \mstar \sim \mstar^{-0.3}$ is accurate, then the
elevated SN Ia rates in dwarf galaxies can be explained by a combination of
their more extended SFHs and an increased binary fraction compared to their
higher mass counterparts due to differences in metallicity.
While~\citet{Gandhi2022} motivate their investigation from this viewpoint, here
we take this argument one step further and postulate that this accounts for
the~\textit{entire} increase in the specific SN Ia rate because the binary
fraction can naturally account for a factor of~$\sim$3 increase over
the~$\sim$1 decade in metallicity spanned by~$\mstar = 10^{7.2} - 10^{10}
\msun$ galaxies.
The suggestion by~\citet{Kistler2013} that the increased WD mass at low
metallicities drives the effect is likely a subdominant effect.
Higher mass WDs for lower metallicity may be an important component if the
steeper~$\sim \mstar^{-0.5}$ scaling inferred with the~\citet{Bell2003} SMF is
correct.
\par
At first glance, an inverse dependence of SN Ia rates on metallicity may seem
at odds with the results in~\citet{Holoien2022} finding that dwarf galaxy
hosts of ASAS-SN SNe Ia tend to be oxygen-rich relative to similar mass
galaxies.
However, SN hosts are likely not a representative sample of the underlying
galaxy population because the steeply declining DTD~\citep[e.g.,][]{Maoz2012a}
means that the intrinsically highest SN Ia rates at any mass should be in
systems which experienced a recent starburst ($\lesssim$1 Gyr ago).
Since oxygen is produced by massive stars with short lifetimes
\citep*[e.g.,][]{Hurley2000, Johnson2019}, these galaxies should also have a
higher-than-average oxygen abundance~\citep[see, e.g.,][]{Johnson2020}.
In other words, SN Ia hosts at fixed mass should be more metal-rich than the
average galaxy.
\par
The calculations we have presented here are simplified in several regards.
We assumed the characteristic SFH predicted by a semi-analytic model of
galaxy formation at all stellar masses.
Our parametrization of the MZR includes no intrinsic scatter, and taking the
\citet{Zahid2014} MZR at face value for use in a power-law scaling implicitly
assumes that all SNe Ia arise from stellar populations near the gas-phase
abundance.
Although in principle galaxies populate distributions of finite width in each
of these quantities, these approximations should be fine for the purposes of
predicting average trends.
\par
Although current surveys lack the depth required to pin down SN rates across
multiple decades of stellar mass at~$z = 1$, the sample sizes necessary to do
so may be available from next-generation facilities.
First and foremost, the Nancy Grace Roman Space Telescope~\citep{Spergel2013,
Spergel2015} will obtain large samples of SNe.
Roman has excellent prospects for discovering all classes of SNe at redshifts
as high as~$z \gtrsim 2$ and beyond~\citep{Petrushevska2016}.
The difficulty in empirically constraining the specific SN Ia rate at
$z \approx 1$ instead comes from uncertainties in the SMF.
Even at~$z = 0$, these measurements are difficult due to the flux-limited
nature of most surveys and the broad range of luminosities and mass-to-light
ratios spanned by galaxies~\citep[see the discussion in][]{Weigel2016}.
Between~$10^{7.2}$ and~$10^{10}~\msun$, the factors of 2 and 3 predicted by our
calculations with~$\gamma = -0.5$ and $\gamma = 0$ are produced by power-law
indices of~$-0.108$ and~$-0.170$, respectively.
The difference between the two (0.062) is the minimum precision required for
the scaling of the SMF at the low-mass end -- only slightly larger than the
precision achieved by~\citet[][$\pm 0.05$, see their Fig. 13]{Baldry2012}.
This empirical test therefore requires at least their level of precision but
at~$z \approx 1$.
\par
A metallicity dependence of~$Z^{-0.5}$ strongly impacts the evolution of Fe
in one-zone models of galactic chemical evolution.
The considerable impact that a~$\gamma = -0.5$ scaling has on the predictions
indicates that evolutionary parameters inferred from one-zone model fits to
multi-element abundance ratios may need revised.
The strongest impact is for dwarf galaxies, where the abundances are low and
the higher yields predict substantial shifts in the position of the
evolutionary track in abundance space.

