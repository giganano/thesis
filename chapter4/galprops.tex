
\section{Galactic Properties}
\label{iarates:sec:galprops}

% fig 1
\afterpage{
\clearpage
\begin{landscape}
\begin{figure*}
\centering
\includegraphics[scale = 0.51]{umachine_sfhs.pdf}
\includegraphics[scale = 0.5]{iarate_vs_tausfh.pdf}
\includegraphics[scale = 0.5]{mzr.pdf}
\caption{
\textbf{Left}: The best-fit mean SFHs of the~\um~galaxies with present-day
stellar masses of $\mstar = 10^{7.5 \pm 0.1}$ (black),~$10^{8.5 \pm 0.1}$ (red),
$10^{9.5 \pm 0.1}$ (green), and $10^{10.5 \pm 0.1} \msun$ (blue) normalized by
their present-day stellar masses.
\textbf{Middle}: The specific SN Ia rate as a function of the e-folding
timescale of the SFH~$\tau_\text{sfh}$ assuming a linear-exponential time
dependence and a~$\tau^{-1}$ power-law SN Ia DTD.
\textbf{Right}: The redshift-dependent MZR reported by~\citet{Zahid2014} at
$z = 0$ (black solid),~$z = 0.5$ (blue), and~$z = 1$ (red).
For comparison, we include the~$z \approx 0$ MZR measured by
\citet[][black dotted]{Andrews2013}.
}
\label{iarates:fig:sfh_mzr}
\end{figure*}
\end{landscape}
\clearpage
}

We begin by examining how the mean galactic SFH varies with present-day stellar
mass as predicted by the~\um~semi-analytic model~\citep{Behroozi2019}.
Using dark matter halo properties supplied by the~\textit{Bolshoi-Planck} and
\textit{Multi-Dark Planck 2} dark matter only simulations~\citep{Klypin2016,
RodriguezPuebla2016},~\um~follows a conventional semi-analytic model framework
(see, e.g., the review in~\citealt{Somerville2015a}) and successfully
reproduces a broad range of well-constrained observables, including stellar
mass functions, cosmic SFRs, specific SFRs, quenched fractions, and UV
luminosity functions.
While some semi-analytic models have used the extended Press-Schechter
formalism~\citep{Press1974, Bond1991} to generate halo merger trees and push
the lower stellar mass limit of their model down to~$\mstar \approx 10^7~\msun$
\citep[e.g.][]{Somerville2015b}, an advantage of~\um~is that the high mass
resolution of the~\textit{Bolshoi-Planck} and~\textit{Multi-Dark Planck 2}
simulations allows merger trees down to~$\mstar = 10^{7.2}~\msun$ to be
obtained directly from the simulations.
Conveniently, this limit is approximately the lowest mass for which there are
empirical constraints on the specific SN Ia rate from ASAS-SN~\citep{Brown2019}
and DES~\citep{Wiseman2021}.
To relate these predictions to data from the untargeted ASAS-SN
survey~\citep{Shappee2014, Kochanek2017}, we take the full galaxy
sample from~\um, including both star forming and quenched galaxies as
well as both centrals and satellites, though centrals are the dominant
population across the full stellar mass range.
\par
In the left panel of Fig.~\ref{iarates:fig:sfh_mzr}, we show the best-fit mean SFH as a
function of lookback time in four narrow bins of present day stellar mass.
In general, low stellar mass galaxies have more extended SFHs than their
higher mass counterparts.
This effect is sufficiently strong that for stellar masses of
$\sim$$10^{7.5}~\msun$, typical SFRs are still increasing at the present day,
while~$\sim$$10^{10.5}~\msun$ galaxies experienced their fastest star formation
long ago.
\par
We adopt a DTD that scales with the age of a stellar population as~$\tau^{-1}$
starting at a delay time~$t_\text{D} = 100$ Myr as suggested by comparisons of
the cosmic SFH with the volumetric SN Ia rate as a function of redshift
(\citealp{Maoz2012a};~\citealp*{Maoz2012b};~\citealp{Graur2013, Graur2014}).
We conducted our analysis using alternative choices of the power-law
index as well as an exponential DTD with an e-folding timescale of
$\tau_\text{Ia} = 1.5$ Gyr and found similar conclusions in all cases.
We do not consider metallicity-dependent variations in the shape of the DTD
here, instead focusing on the overall normalization.
In principle, the minimum delay of the DTD could be as short as~$\sim$40 Myr if
WDs are produced by~$\lesssim$8~\msun~stars~\citep*[e.g.,][]{Hurley2000}, and
perhaps even shorter at low metallicity if the total metal content of a star
significantly impacts its lifetime~\citep[e.g.,][]{Kodama1997, Vincenzo2016a}.
However, if SNe Ia require some additional time following WD formation, the
minimum delay will be longer.
Since we are interested in the first-order effects of variations in the SFH on
specific SN Ia rates, we assume a value of~$t_\text{D} = 100$ Myr.
In calculations using both~$t_\text{D} = 40$ Myr and~$t_\text{D} = 150$ Myr,
we found similar results.
\par
For an SFH~$\dot{M}_\star$ and DTD~$R_\text{Ia}$ as functions of lookback time
$\tau$, the specific SN Ia rate at a stellar mass~$\mstar$ is
\begin{equation}
\frac{\dot{N}_\text{Ia}(M_\star | \gamma)}{M_\star} \propto Z(M_\star)^\gamma
\ddfrac{
	\int_0^{T - t_\text{D}}\dot{M}_\star(\tau | M_\star) R_\text{Ia}(\tau) d\tau
}{
	\int_0^T \dot{M}_\star(\tau | M_\star) d\tau
}
\label{iarates:eq:specia}
\end{equation}
where~$T = 13.2$ Gyr is the time elapsed between the onset of star formation
and the present day.
To investigate the effects of metallicity, we add a power-law metallicity
scaling~$Z(\mstar)^\gamma$ where~$Z$ is given by the MZR.
We are only interested in the scaling of the rates with~\mstar, so we normalize
all rates to unity at~$\mstar = 10^{10}~\msun$ following~\citet{Brown2019}.
Although the denominator of equation~\refp{iarates:eq:specia} in detail should depend
on mass loss from stars as they eject their envelopes, this is an approximately
constant term which can safely be neglected in the interest of computing
relative rates ($\approx$40\% for a~\citealt{Kroupa2001} IMF; see discussion
in~\S\S~2.2 and 3.7 of~\citealt*{Weinberg2017b}).
\par
To qualitatively illustrate how the specific SN Ia rate scales with the
timescale over which star formation occurs, we consider the simple example of a
linear-exponential
parametrization~$\dot{M}_\star \propto te^{-t/\tau_\text{sfh}}$ where
$t = T - \tau$.
The middle panel of Fig.~\ref{iarates:fig:sfh_mzr} shows equation~\refp{iarates:eq:specia} as
a function of the e-folding timescale~$\tau_\text{sfh}$ assuming~$\gamma = 0$.
The specific SN Ia rate is lowest in the limiting case of a single episode of
star formation (i.e.,~$\tau_\text{sfh} \rightarrow 0$), rises steeply until
$\tau_\text{sfh} \approx 10$ Gyr, and then flattens once
$\tau_\text{sfh} \gtrsim T$.
A higher specific SN Ia rate as observed in dwarf galaxies is therefore a
natural consequence of their more extended SFHs, though we demonstrate below
that this effect accounts for only a factor of~$\sim$2 increase in the rate
between~$10^{7.2}$ and~$10^{10} \msun$.
\par
The right panel of Fig.~\ref{iarates:fig:sfh_mzr} shows the MZR parametrized by
\citet[][see their equation 5]{Zahid2014}\footnote{
	We have transformed from their~$\log_{10}\text{(O/H)}$ measurements to the
	logarithmic abundance relative to the Sun~$\log_{10}(Z / Z_\odot)$ assuming
	the Solar oxygen abundance derived by~\citet{Asplund2009}.
} at redshifts~$z = 0$, 0.5 and 1 in comparison to the~\citet{Andrews2013}
parametrization at~$z = 0$.
Although~\um~allows us to investigate these effects at stellar masses as low as
$10^{7.2}~\msun$, the~\citet{Zahid2014} measurements are available only for
$\mstar \approx 10^9 - 10^{11}~\msun$ galaxies.
\citet{Andrews2013} used stacked spectra from the Sloan Digital Sky Survey
\citep[SDSS;][]{York2000} to obtain direct measurements of the oxygen abundance
in bins of stellar mass extending as low as~$\sim$$10^{7.4}~\msun$.
Relative to~\citet{Zahid2014}, the~\citet{Andrews2013} parametrization has a
lower plateau but otherwise a similar slope and turnover mass.
Because we simply normalize the rates to unity at~$\mstar = 10^{10}~\msun$,
only the shape of the MZR matters, and we find similar results using both
parametrizations.
In order to estimate SN Ia rates at redshifts of~$z = 0.5$ and~$z = 1$,
we use the redshift-dependent~\citet{Zahid2014} formalism
in~\S~\ref{iarates:sec:predictions}.
\par
Given a present-day stellar mass, we compute its SFH as a function of
lookback time by interpolating between the stellar mass and snapshot times
included in the~\um~predictions.
We then compute the specific SN Ia rate according to equation~\refp{iarates:eq:specia}
given the implied SFH and and a~$\tau^{-1}$ DTD, amplifying the rate by a
factor of~$Z^\gamma$ where the metallicity~$Z$ is computed from the
\citet{Zahid2014} MZR.
Because these calculations are simply using the~\um~SFHs, the results are
unaffected by the SMF dependence of the observational estimates (i.e.,
Eq.~\ref{iarates:eq:specia} can simply be divided by~$\mstar$ as opposed to an integral
over the SMF).

