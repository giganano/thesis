
\documentclass[main.tex]{subfiles}
\begin{document}

\chapter{\texttt{VICE}} 
\label{migration:sec:vice} 

\vice~is an open-source~\texttt{python} package available for Linux and Mac OS 
X.\footnote{ 
	Install (PyPI): \url{https://pypi.org/project/vice} \\ 
	Documentation: \url{https://vice-astro.readthedocs.io} \\ 
	Source Code: \url{https://github.com/giganano/VICE.git} 
} 
Windows users should install and use~\vice~entirely within the Windows Subsystem 
for Linux. 
The latest version (1.2.1) requires~\texttt{python >=}~3.6 and can be 
installed in a terminal via~\texttt{pip install vice}, after 
which~\texttt{vice -{}-docs} will launch a web browser to the documentation 
at~\url{https://vice-astro.readthedocs.io}. 
\texttt{vice -{}-tutorial} will also 
launch a web browser, but to a jupyter notebook in the GitHub repository 
intended to familiarize first-time users with~\vice's API. 
\texttt{Python} code that runs the simulations presented in this paper is 
included as supplementary material in the GitHub repository; they can be run 
from a terminal without modifying the source code. 
When taking into account time-dependent stellar migration,~\vice~requires 
$\sim$2.5 CPU-hours per chemical element to compute masses, abundances, and 
initial and final Galactic regions for~$\sim$1.6 million stellar populations 
spanning a little over 1,300 timesteps. 
This estimate was made using a single core with a 3 GHz processor. 
Although their outputs require only~$\sim$235 MB of disc space each, our models 
as computed in this paper can require up to~$\sim$3 GB of RAM at any given 
time owing to the number of timesteps and stellar populations used, but these 
can be adjusted via command line arguments. 
Beyond what has been presented in this paper,~\vice's capabilities include 
user-defined, arbitrary functions of time describing star formation and infall 
histories, star formation laws, outflow prescriptions, SN Ia delay-time 
distributions, and element-by-element infall metallicities. 
It allows the IMF to be a user-defined function of stellar mass. 
It will compute yields from supernova and asymptotic giant branch star 
nucleosynthesis studies, but allows the user to specify arbitrary mathematical 
forms for use in chemical evolution models. 
A complete breakdown of its abilities can be found in the documentation. 
While providing this level of versatility in a~\texttt{python} package, 
\vice~also enjoys a backend implemented in ANSI/ISO~\texttt{C}, providing it 
with the powerful computing speeds of a compiled library; with typical 
parameters, yield calculations and one-zone model integrations require only a 
fraction of a second. 

\end{document}

