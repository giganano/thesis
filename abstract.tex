
\documentclass[main.tex]{subfiles}
\begin{document}

\addcontentsline{toc}{chapter}{Abstract}
\null\par\null\par\null\par\null\par
\begin{center}
{\large Abstract}
\end{center}

\begin{doublespace}

The chemical composition of the universe is constantly changing.
With only hydrogen, helium, and trace amounts of lithium left over in the wake
of the Big Bang, all heavier atomic nuclei in the universe were produced
through the fusion of lighter nuclei inside stars.
When a star dies, it disperses a considerable portion of this material back to
its surroundings.
As the sites of star formation in the universe, galaxies are like petri
dishes of their own nuclear reactions, facilitating the formation of new
stars and retaining the heavy nuclei they produce.
By analyzing the chemical abundance structure of galaxies, we can deduce both
their evolutionary histories and the stellar evolution processes which
produced the stable elements on the periodic table.
\par
Thanks to the advent of large spectroscopic surveys, this field
of~\textit{galactic archaeology} has recently ushered in a new age.
The APOGEE survey alone has estimated abundances of at least 15 different
elements in over 650,000 stars in the Galaxy.
In this dissertation, I draw on galactic chemical evolution (GCE) models to
shed light not only the processes shaping galaxy evolution, but also the
mechanisms of nucleosynthesis in stars.
\par
Using a powerful and efficient GCE software developed as part of this work,
I quantify the impact of sudden bursts in star formation in dwarf galaxies and
develop methods with which to pin down the details of these events and the
evolutionary timescales at play.
Introducing elemental yields as free parameters, I demonstrate that this
framework can deduce the evolutionary histories of galaxies and yeilds from
stellar populations simultaneously.
Applications of this methodology to disrupted dwarf galaxies in the Milky Way's
stellar halo observed with the H3 survey provide results consistent with
known trends of galaxy properties with stellar mass.
\par
To harness the constraining power of supernova surveys in GCE models, I
investigate the origin of the observed high Type Ia supernova rates in dwarf
galaxies.
With both a lower metal content than their higher mass counterparts and more
extended star formation histories, both of these effects are required to
make sense of the results.
The increased prevalence of close binary stars at low metallicities lends a
natural explanation to the origin of the metallicity dependence of the rate.
These results have strong implications for the enrichment histories of galaxies
by directly increasing the rate of metal production at low abundances.
\par
Harnessing the power of APOGEE, I then assess which observed characteristics of
the disk abundance structure in the Milky Way can be described by classic
inside-out Galaxy growth and the radial migration of stars.
This empirically motivated combination successfully reproduces many of the
observed correlations between stellar age and chemical compositions.
Its most interesting failure is its inability to explain the observed dichotomy
in multi-element abundances across the disk.
This result suggests that the Galaxy experienced a rather episodic star
formation history.
\par
Using these models of the Milky Way as a prototype, I turn them toward similar
external galaxies at low redshift.
I further demonstrate the ability of the GCE framework to deduce elemental
yields from stars by comparing simple parameterizations of nitrogen yields from
stars against a compilation of measurements of the seemingly universal trend
between gas-phase nitrogen and oxygen abundances.
I find that the nitrogen yield relative to the oxygen yield must increase
approximately linearly with the metal mass fraction with a constant floor at
low metallicity, a result which is quite insensitive to variations in, e.g.,
the star formation history and efficiency of outflows from the disk.
This result has strong implications for stellar evolution due to the sensitivity
of nitrogen production to poorly understood processes in stars.
\par
Lastly, I quantify the evolution of the radial metallicity gradient in APOGEE
by conditioning a large sample of stars on their ages.
I find that a gradient closely resembling that of the youngest stars in the
Galaxy was established~$\sim$9 Gyr ago.
This result is best explained by GCE models invoking equilibrium abundances,
whereby metal production is balanced by losses to, e.g., new star formation and
outflows.
This explanation contrasts with the traditional view that gradients are a
consequence of inside-out Galaxy growth.
The age independence of the stellar metallicity gradient and its interpretation
as a consequence of equilibrium reframe the discourse surrounding abundance
gradients in spiral galaxies.

\end{doublespace}

\end{document}

