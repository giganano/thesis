
\documentclass[main.tex]{subfiles}
\begin{document}

\chapter{Conclusions}
\label{conclusions}

In this dissertation, I have used GCE models to deepen our understanding of
galaxy evolution across a broad range of stellar mass.
Through numerical modeling, I have characterized the observational
signatures of distinct evolutionary pathways in galaxies across a broad range
of stellar mass and developed statistical methods with which to quantify these
processes.
By introducing stellar yields as free parameters, these investigations have
the ability to deduce elemental yields alongside galaxy evolution parameters.
The large samples of multi-element abundances provided by modern spectroscopic
surveys have been invaluable empirical test beds.

\section{Dwarf Galaxies}
\label{conclusions:sec:dwarfs}
\textbf{Sudden Events:} I have investigated the expected chemical signatures
from sudden events in the evolutionary histories of dwarf galaxies.
Due to a perturbation in the ratio of Type Ia and Type II supernova rates,
the principal signature of a burst in star formation is a population of
stars with super-solar [O/Fe] ratios.
If the decay of the starburst is sufficiently fast, then the inverse effect can
also arise, whereby a population of~\textit{sub-solar} [O/Fe] stars is left
behind.
Through this investigation, I also developed a high-performance GCE software,
which was further developed in later chapters and ultimately became the
backbone of much of this dissertation's analysis.
\par
\textbf{Dwarf Galaxy Archaeology:} I have applied statistical regressions to
abundance models to provide a means with which to pin down these events and
evolutionary timescales quantitatively.
After a rigorous validation of the methodology, I have applied this method to
two galaxies of considerably different masses (a factor of~$\sim$50), finding
results consistent with known trends in galaxy mass.
This methodology can simultaneously deduce the relative yields of different
elements required to explain the results.
\par
\textbf{Type Ia Supernova Rates:} I have investigated the origin of the high
Type Ia supernova rates observed in dwarf galaxies.
I found that mass is simply a confounding variable, and that perhaps the more
fundamental correlation is the metal abundance and the supernova rate.
Low stellar mass galaxies are known empirically to host lower metallicity
stellar populations~\citep[e.g.,][]{Gallazzi2005, Kirby2013}.
These populations have a higher close binary fraction than their high
metallicity counterparts~\citep[e.g.,][]{Badenes2018, Moe2019}, increasing the
frequency of potential Type Ia supernova progenitors.
These results have strong implications for GCE models through a significant
increase in enrichment rates at low metallicities.

\section{The Milky Way}
\label{conclusions:sec:milkyway}
\textbf{Stellar Migration:} I have found that inside-out Galaxy growth
\citep[e.g.,][]{Bird2013} combined with stellar migration~\citep{Sellwood2002}
lends a natural explanation to some but not all of the observed characteristics
of the disk abundance structure.
Many correlations between stellar age and chemical composition
\citep[e.g.,][]{Feuillet2019} are naturally explained in this manner.
Anomalies in these relationships, such as young stars with super-solar [O/Fe]
ratios~\citep[e.g.,][]{Hekker2019} also arise as an indirect consequence of
orbital migration.
The most interesting failure of this model is its inability to reproduce the
chemical dichotomy seen in disk stars~\citep[e.g.,][]{Hayden2015}, indicating
that the Galaxy had a more episodic evolutionary history
\citep[e.g.,][]{Chiappini1997, Spitoni2021} and rebutting models which claim
this combination to be a sufficient explanation of the results
\citep[e.g.,][]{Sharma2021, Chen2023}.
In the process of this work, I significantly expanded upon the capabilities
provided by my GCE software, catalyzing further investigations of the Milky
Ways assembly history.
\par
\textbf{Nitrogen:} I have applied the GCE model of Chapter~\ref{migration} to
the enrichment of nitrogen.
A unique element from a nucleosynthesis perspective, theoretical predictions
of nitrogen yields from stellar populations are significantly discrepant with
one another.
The differences arise because nucleosynthetic yields are sensitive to a
variety of processes that are poorly understood, for which there are a variety
of assumptions in published models (see discussion in,
e.g.,~\citealt{Chieffi2001, Ventura2013}).
In the case of N specifically, if too wide of a mass range of asymptotic giant
branch stars experience both third dredge up and hot bottom burning, then the
metallicity dependence of the yield inverts sign, resulting in empirically
untenable abundance trends.
This realization has significant implications for stellar evolution models.
\par
\textbf{Metallicity Gradients:} Ubiquitous in spiral galaxies like the Milky
Way, radial metallicity gradients have classically been interpreted as a
consequence of inside-out disk growth~\citep[e.g.,][]{Kauffmann1996}.
I have constrained the origin of the gradient by conditioning stars on their
ages as estimated in APOGEE~\citep{Majewski2017, Mackereth2019b}.
I found that a gradient closely resembling that of the present day was
established as long ago as~$\sim$9 Gyr.
I demonstrated that this result is best explained by GCE models which invoke
chemical equilibrium, though the detailed cause and origin of what drove the
Galaxy to equilibrium so quickly remains unknown.
An increase in the strength of mass loading with radius is a plausible
explanation, though it is only one.
An indication that inside-out growth only plays a role in the earliest epochs
of disk formation, this comparison marks a significant shift in the discourse
surrounding abundance gradients.

\section{Future Work}
\label{conclusions:sec:future-work}

The show must go on.
Galaxy evolution and nucleosynthesis are far from solved problems.
The tools available now and in the near future have the power to illuminate
answers to these questions about our place in the universe.
I am delighted to be continuing my work as a postdoctoral fellow at The
Observatories of the Carnegie Institution for Science in Pasadena, California.
\par
In collaboration with Ana Bonaca, I will expand my investigations into the
evolutionary histories of dwarf galaxies.
We will combine the GCE regressions of Chapter~\ref{dga} with color-magnitude
diagram analysis~\citep[e.g.,][]{Dolphin2002, Weisz2014a} to constrain the age
distributions of stars too distant for individual age estimates.
I will use the combined powers of H3~\citep{Conroy2019} and the upcoming
SDSS-V~\citep{Kollmeier2017} to model multi-element abundance ratios of a
suite of accreted dwarf galaxies in the stellar halo.
Also in collaboration with Josh Simon, we will apply the same analysis to
local group dwarf galaxies where resolved stars are accessible with the
Magellan InfraRed Multi-Object Spectrograph~\citep[MIRMOS;][]{Konidaris2022} on
the Magellan Telescopes at Las Campanas Observatory.
The disrupted and intact dwarf galaxy samples combined, we will have conducted
the most detailed ``chemical census'' of local group dwarf galaxies to date.
\par
I will continue my investigations into the enigma that is our own Galaxy.
I am excited to collaborate with Andrew Benson to use the large sample that
will be provided by SDSS-V's Milky Way Mapper program to distinguish
quantitatively between enrichment models of the Milky Way.
I am particularly interested in understanding the origin of the chemical
dichotomy in the disk~\citep[e.g.,][]{Hayden2015}, which is the most
interesting failure of the models in Chapter~\ref{migration}.
Though the sample size of large spiral galaxies with resolved stars will
remain small, upcoming observations of red giants in the disk of Andromeda with
the James Webb Space Telescope will provide an enightening comparison case to
understand the ways in which the Milky Way might be unique.
\par
I will continue my research program empirically calibrating yields of various
elements against abundance ratio trends using GCE models.
I am interested in working with Andrew McWilliam to constrain the nature of
Type Ia supernova explosions.
Manganese has identified as a particularly useful element as its yield is
predicted to be highly sensitive to the density, and by extension the mass,
of the exploding white dwarf~\citep[e.g.,][]{delosReyes2020}.
I am also excited to work with Rebecca Bernstein on identifying the
astrophysical origins of rapid neutron capture nucleosynthesis.
The wealth of multi-element abundance measurements that will be available
through Milky Way Mapper will provide excellent empirical diagnostics for both
investigations.
\par
In addition to these ambitious research programs, I plan to maintain my efforts
in the mentoring of undergraduate students through the Carnegie Astrophysics
Summer Student Internship (CASSI) program.
Should the opportunity arise, I also look forward to expanding these efforts
to help advise students at the graduate level.
As a member of the leadership committee of Polaris, a student-led organization
at Ohio State dedicated to improving the retention of students from
disadvantaged backgrounds in physics and astronomy, I am delighted that CASSI
sees considerable participation from underrepresented minority groups.
I also look forward to getting involved in Carnegie Observatories' various
outreach efforts and open houses to engage with the local community.

\end{document}

